\documentclass[conference]{IEEEtran}

\usepackage{graphicx}
\usepackage{subfigure}


\hyphenation{op-tical net-works semi-conduc-tor}


\begin{document}
%
% paper title
% can use linebreaks \\ within to get better formatting as desired
\title{Better JavaScript Runtime Understanding by Automatic Function Naming}


% author names and affiliations
% use a multiple column layout for up to three different
% affiliations
\author{\IEEEauthorblockN{Salman Mirghasemi}
\IEEEauthorblockA{\'Ecole Polytechnique F\'ed\'erale de Lausanne\\
Lausanne, Switzerland\\
salman.mirghasemi@epfl.ch}
\and
\IEEEauthorblockN{John J. Barton}
\IEEEauthorblockA{IBM Research - Almaden\\
San Jose, USA\\
johnjbarton@johnjbarton.com}
\and
\IEEEauthorblockN{Claude Petitpierre}
\IEEEauthorblockA{\'Ecole Polytechnique F\'ed\'erale de Lausanne\\
Lausanne, Switzerland\\
claude.petitpierre@epfl.ch}}
\maketitle


\begin{abstract}
%\boldmath
Understanding JavaScript code due to its dynamic, weakly-typed nature is complicated. Developers usually understand JavaScript programs by running them and examining their elements at runtime. However, understanding the concrete value of an element is not always straightforward. Among all runtime elments in JavaScript functions are more important: they are used as object constructors and apear in call stack. We discuss the function names usage in improving concrete values presentation. Morover, we propose an approach for automatic function naming in JavaScript.
\end{abstract}

\section{Introduction}
The unique and important role of JavaScript in web programming is undeniable. This language is used by 97 out of the web's 100 most popular sites\footnote[1]{http://www.alexa.com}. It is very likely that JavaScript keeps this crucial role at least for the next few years. Along with the growth of demands for more comprehensive user interfaces, the size and the complexity of web applications is increasing. Morover, JavaScript is also becoming a general purpose computing platform with office applications \cite{JSOffice, JSOffice2}, browsers \cite{FAO, GCE} and development environments \cite{Ingalls} being developed in JavaScript \cite{Richards}. There are also proposals for employing JavaScript in server-side applications \cite{SSJSR, CJS}.

To cope with the pace of changes, JavaScript developers need to improve their develoment processes and practices. They need modern editors for writing and editing larger programs and effective tools for understanding complicated JavaScript runtime. However, there has been severe inherent barriers for improving JavaScript tool support. 
Unlike many traditional object oriented languages such as Java and C\#, it does not have classes, and does not encourage encapsulation or even structured programming JavaScript is a weakly typed language with no type declarations and only run-time checking of calls and field accesses \cite{Richards}. As a direct consequence, the JavaScript code contains less explanatory data about the program elements and their relations. A variable value can be a primitive value, an object with any structure or even a function. One callsite may invoke different function bodies with different number of arities that none of them can be understood from the code. Therefore, the lack of descriptive data in JavaScript code is the most fundumental issue in enhancing JavaScript tool support.

Catching errors early (i.e., before or at the compilation time) is an important feature for development environments. It can save developers' time, prevent bugs and improve the program reliability. However, many errors can not be recognized without a strong typing stystem. To attack this issue a few static typing systems have been proposed for JavaScript \cite{Anderson, Anderson2, Heidegger, Thiemann}. These approaches discover the type of values and object structures for a variable by statically analyzing the source code and possible program control flows lead to the variable assignment. These discovered facts about variable types are not only useful for catching errors but to provide modern editing features such as auto-complete and refactoring in development environments. Moreover, these infered data can help in code coprehension if they are properly presented. However, none of the mentioned approaches provided effective means for sharing this information with the developer. 

%talk about runtime tools
Developers usually understand JavaScript code by examining the running program in debuggers. At runtime, concrete values are available and developer can directly check and understand object structures and control flows. However, understanding a concrete value is not always straightforward, particulary for non-primitive values such as objects and functions. To facilitate understanding these complex values, debuggers show a summary of these objects. For example, the name of object constructor must appear in its summary.

Functions are central to program comprehension in JavaScript. They are first-class objects that can be used for object construction and be passed as function call arguments. However, there is a serious issue with JavaScript functions that makes understanding them very difficult. Functions can be defined and created with no identifier or name. A function name is important for referring and recalling the function's functionality. Morover, the name of an object's constructor can work like a class name and summarizes the object structure. 

In this paper, we propose an approach for automatic naming of functions in JavaScript. The function names can be used in debuggers for more descriptive object summaries or callstack view. Moreover, these function names can be used in integration with proposed JavaScript typing systems for providing modern editing features.

%Many developers develop JavaScript in object-oriented way.
%

%For example to know an object, developer has to examine the object properties and from the property names and values, some connections
%understanding JavaScript code and its runtime is much harder for developers.
% we propose automatic function naming with these features
% 1) expected by user 2) different situations

\begin{figure*}[htp]
\centerline{
\subfigure[Firefox Firebug Debugger]{\label{fig_first_case}\includegraphics[width=.47\textwidth]{fbug-callstack.jpg}}
\hfil
\subfigure[Google Chrome Debugger]{\label{fig_second_case}\includegraphics[width=0.47\textwidth]{chrome-callstack.jpg}}}
\caption{These two pictures show the call stack view of Firebug and Google Chrome JavaScript debuggers paused at the same point. 
The Google Chrome debugger is unable to find three function names and The Firebug debugger fails at one case.}
\label{debuggers}
\end{figure*}

\section{The Function Naming Problem}
Figure~\ref{debuggers} shows the call stack view of Firefox Firebug and Google Chrome JavaScript debuggers, paused at the same breakpoint. The differences between two call stack, particularly in the first two frames, are due to different implementations of event handling in the underlying platforms. Google chrome shows \texttt{anonymouse} for three functions which doesn't give any useful information to the developer. The developer has to look at the specified source file and line number to recall the function. Although Firebug shows more function names, it is also unsuccessful about one function.
 	
%___ The importance of having function names
The importance of function names is not only limited to call stack view. A function can be used as a constructor for constructing new objects. The constructor name like class name in traditional languages such as Java, can be used in object representation. It can assist developers in understanding the structure and role of the object without looking into its properties. 

In JavaScript, defining \textit{identifier} for functions is optional. A function can be assigned to a variable or an object property with any name. Therefore a function object can be called by different names during in life time. 
%___ why it is hard

%Many developers develop JavaScript in object-oriented way.
%


%___ related work for type system

%-- Check here: http://kangax.github.com/nfe/
%function body(source), function object

%explain function expression and function declerations
%moreover explain creation of the list of outer scopes.
In JavaScrip a function can be defined by a function expression or a function decleration \cite{ECMA}. From the same function expression or decleration, function objects are created. In addition to function source code, the clousure scopes are also important for understanding the function operation. The clousure scopes include all available variables to the function at the time of creation. The location of function definition in the source code is enough for knowing the clousure scopes. Therefore, a function body with its location in a source code defines a category of function objects with the similar behaviour.
Is it possible to assing a name to every function object in JavaScript in a way that:
1) Function objects with the same code has the same name.
2) The proposed name recalls the function for the developer.


\section{Automatic Funcation Naming}
To choose the most appropriate name for a function which has the characteristics defined in the previous section, we consider two cases. First, assigning a global name to the function which identifies the function in the whole program. Second, naming a function in a call stack.


\subsection{A Global Name for a Function Object}

\begin{figure}[htp]
\begin{verbatim}
(a) Case 1:
 function foo(){ ... }

(b) Case 2:
 var foo = function { ... } 
 
(c) Case 3:
 var foo = {
     bar : function { ...}
 }

(d) Case 4:

 var foo = function(){
           ....
           return function(){ ... }
 }()
 
(e) Case 5:
 var foo = function(){ 
 					 return ...;
 }()					
 
(f) Case 6:
 var foo = function(){
 
 }()();
 
(g)
 foo(function(){ ... });
 
(h)
 array = [...., function(){},...]
 
(i)
 array[i] = function(){};   

\end{verbatim}
\caption{Different cases of function object creation.}
\label{fig:functionCreation}
\end{figure}

\subsection{Naming Functions in Call Stack}



\begin{figure}[htp]
\begin{verbatim}
(a)
 foo();
(b)
 foo.bar();
(c)
 foo.apply(bar, args);

\end{verbatim}
\caption{Different cases of function object creation.}
\label{fig:functionCall}
\end{figure}

\section{Related Work}
%Another strand of research has tried to investigate how to provide better tools for developers for catching errors early. Being a weakly typed language with no type declarations and only run-time checking of calls and field accesses, it is natural to try to provide a static type system for JavaScript [2, 1, 3, 24, 13]. Finally, after many years of neglect, modern implementations of JavaScript have started to appear which use state of the art just-in-time compilation techniques [10].

%For JavaScript, Anderson et al. proposed a type system with definite and potential types [2, 1, 3], while Heidegger and Thiemann following up on some of their earlier work [24, 18] propose recency types in [13], and Furr et al. proposed a related system for DRuby [9]. While all of these type systems acknowledge some minor simplifications to the target language, they rely on fairly similar assumptions. For instance, Thiemann writes: �Usually, no further properties are defined after the initialization and the type of the properties rarely changes.� This suggests that object types are stable at run-time and can be described using, e.g., traditional rowtypes. In fact all the proposals take the view that an object�s type should be the sum of all possible fields and methods that it could contain, with some of them being undefined; they differ mostly on how to perform strong updates to avoid polluting all properties with undefined values. Interestingly, language implementors make similar assumptions. For instance, Google�s V8 JavaScript engine is reported to optimistically associate �classes� to objects on the assumption that their shape will not change too much, though with a fallback case for highly dynamic objects 3. This design is similar to implementations of one of JavaScript�s influences, Self [5], and is expected to work for the same reasons. As the above mentioned hypothesis is crucial for the applicability and usefulness of the results, it deserves careful study. In fact, we have found a number of similar assumptions in the literature which we list below.We first review the salient features of the language to provide sufficient background for readers unfamiliar with JavaScript.

\section{Conclusion}
In this paper we presented an approach for naming JavaScript functions both
as individual objects and in call stack. 


\begin{thebibliography}{10}

\bibitem{Anderson}
C. Anderson, P. Giannini, and S. Drossopoulou. \newblock Towards Type Inference for JavaScript.
\newblock In \emph{Proceedings of the 19th European conference on Object-Oriented Programming(ECOOP)},
July, 2005.

\bibitem{Anderson2}
C. Anderson and P. Giannini. \newblock Type checking for javascript.
\newblock \emph{Electr. Notes Theor. Comput. Sci.}, 138(2), 2005. 

\bibitem{CJS}
Common JS.
\newblock http://www.commonjs.org/

\bibitem{ECMA}
ECMA International.
\newblock \emph{ECMA-262: ECMAScript Language Specification},
ECMA (European Association for Standardizing Information
and Communication Systems), Geneva, Switzerland, third edition,
December 1999. 

\bibitem{FAO}
Firefox Add-ons.
\newblock https://addons.mozilla.org/en-US/developers/docs/gett ing-started.

\bibitem{GCE}
Google Chrome Extensions.
\newblock http://code.google.com/chrome/extensions.

\bibitem{Heidegger}
P. Heidegger and P. Thiemann.\newblock Recency types for dynamically-typed, object-based languages.
\newblock In \emph{Proceedings of Foundations of Object Oriented Languages (FOOL)},
2009.

\bibitem{Ingalls}
D. Ingalls, K. Palacz, S. Uhler and A. Taivalsaari.\newblock The lively kernel a self-supporting system on
a web page.
\newblock In \emph{Self-Sustaining Systems},
2008.

\bibitem{JSOffice}
JScript development in Microsoft Office 11.
\newblock http://msdn.microsoft.com/ en-us/library/aa202668(office.11).aspx

\bibitem{JSOffice2}
JavaScript development in OpenOffice.
\newblock http://framework.openoffice.org/ scripting/release-0.2/javascript-devguide.html

\bibitem{Richards}
G. Richards, S. Lebresne, B. Burg and J. Vitek.\newblock An analysis of the dynamic behavior of JavaScript programs.
\newblock In \emph{Proceedings of the 2010 ACM SIGPLAN conference on Programming language design and implementation(PLDI)},
June, 2010.

\bibitem{SSJSR}
Server-Side JavaScript Reference v1.2.
\newblock http://research.nihonsoft.org/ javascript/ServerReferenceJS12.

\bibitem{Thiemann}
P. Thiemann.\newblock Towards a type system for analyzing JavaScript programs.
\newblock In \emph{Proceedings of European Symposium on Programming (ESOP)},
2005.

 
\end{thebibliography}



% that's all folks
\end{document}


