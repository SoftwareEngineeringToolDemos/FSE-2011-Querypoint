
%-----------------------------------------------------------------------------
%
%               Template for sigplanconf LaTeX Class
%
% Name:         sigplanconf-template.tex
%
% Purpose:      A template for sigplanconf.cls, which is a LaTeX 2e class
%               file for SIGPLAN conference proceedings.
%
% Author:       Paul C. Anagnostopoulos
%               Windfall Software
%               978 371-2316
%               paul@windfall.com
%
% Created:      15 February 2005
%
%-----------------------------------------------------------------------------


\documentclass[preprint]{sigplanconf}

% The following \documentclass options may be useful:
%
% 10pt          To set in 10-point type instead of 9-point.
% 11pt          To set in 11-point type instead of 9-point.
% authoryear    To obtain author/year citation style instead of numeric.

\usepackage{amsmath}
\usepackage{graphicx}
\usepackage{subfigure}


\begin{document}

\conferenceinfo{PLDI'11}{June 4-8, San Jose, California.} 
\copyrightyear{2011} 
\copyrightdata{[to be supplied]} 

\titlebanner{preprint}        % These are ignored unless
\preprintfooter{}   % 'preprint' option specified.

\title{Debugging by \textit{lastChange}}
\subtitle{}

\authorinfo{Salman Mirghasemi\and Claude Petitpierre}
           {Ecole Polytechnique F\'ed\'erale de Lausanne}
           {\{salman.mirghasemi,claude.petitpierre\}@epfl.ch}

\authorinfo{John J. Barton}
           {IBM Research - Almaden}
           {johnjbarton@johnjbarton.com}

\maketitle

\begin{abstract}
Backward search from bug symptoms to defects-which cause the bug-is a natural way to debugging. However, due to the lack of support for easily bakward movement by  traditional debugging approaches (e.g., breakpoint-based or log-based debugging), applying this strategy-if it is not pracatically impossible-demands a lot of effort from developers. Tracking origins to wrong values is a fundumental step in this process and any aid in this regard can greatly reduce the complexity of debugging for developers.

To attack the mentioned problem, we propose a new functionality in debuggers, \textit{lastChange}, which locates the last place a value is changed and lets developer to collect data and ask for more \textit{lastChange} queries at the located point. While \textit{lastChange} algorithm is based on program re-execution, contrary to other replay-based algorithms which require exactly the same re-executions, it only requires \textit{bug reproduciblity} (i.e., a test case is available which reproduces the bug).

As a proof of concept we developed \textit{RevNav}, a JavaScript prototype, which adds the \textit{lastChange} feature to Firebug JavaScript debugger. Moreover, \textit{RevNav} provides mechanism for automated bug reproduction, and a centralized user interface which summarizes investigated execution points and collected results.

% how we buil it on breakpoint technology
\end{abstract}

\category{D.2.5}{Testing and Debugging}{Debugging aids}
\category{D.2.6}{Programming Environments}{Integrated environments}

\terms
Algorithms, Human Factors, Languages

\keywords
LastChange, Locating Defects, Breakpoint, Watchpoint, Logging

%---------------------------------------------------------------------------------------------------
\section{Introduction}
Debugging is an inevitable part of programming, still hard and time-consuming. To fix a bug, developers have to reproduce and monitor the buggy execution several times to understand the program's unexpected behavior. Trial and error, guess-work and analyzing huge collected data are the inseparable parts of this process. According to \cite{LaToza}, developers spend about fifty percent of their time debugging. Therefore, every improvement to debugging techniques and tools can considerably save developers' time and improve programs' quality.

Locating defects cause the bug is the main part of debugging. A common strategy for locating defects is starting from bug symptoms and backward movement by following the origings to wrong values. While this strategy seems straight-forward and effective, applying this strategy using traditional methods is complicated. There are two traditional approaches to debugging, breakpoint-based and log-based debugging. None of these approaches assist developer in finding origings to a wrong value. It is due to developer to search source files, find the list of possible origings to a wrong value and set breakpoints or insert log statements in a way that covers all possible origins.

The next dilemma is data collection. In breakpoint debugging, developer has to memorize values or collect data manually at every breakpoint hit and in cases that there are many breakpoint hits, this becomes a tedious task. In log-based debugging, developer has to decide about data should be collected when inserts the log statement. It is very common that the data developer decided to be collected is insufficient. Developer has to change the log statements and re-executes the program another time. Once the sufficient data is collected, it still requires analysis and understanding. Developer usually ends up in dealing with long log files and analyzing huge collected data.

Our proposal is a new functionality in debuggers which locates the origin of a wrong value, call it \textit{lastChange}. The only prerequisite for \textit{lastChange} functionality is bug reproduciblity (i.e., there is a test case which reproduces the bug). Assume that the program execution is paused on a breakpoint and the developer is suspicious about the value of an object property or variable. The developer selects \textit{lastChange} on the value and it is due to debugger to locate and shows the last change of object property or variable. Debugger reproduces the buggy execution and collects limited data and once the execution reaches the same place (i.e., the same breakpoint hit), call it \textit{reproduction point}, it pauses the execution, analyzes the collected data and shows the location of the last change to the developer. Developer can also examine the program state at the located point of execution, and ask for a new \textit{lastChange} at this point.

Our contribution in this paper is the algorightm \textit{lastchange}, which locates the last place a value is changed, gathers other values from that execution point, and allows \textit{lastChange} operations from that point. The algorithm builds on existing breakpoint debugger technology. We demonstrate the feasibiltily of the algorithm with \textit{RevNav}, an implementation extending the Firebug JavaScript debugger. \textit{RevNav} also provides mechanism for automated bug reproduction, and a centralized user interface which summarizes investigated execution points and collected results. 

The rest of the paper is organized as follows. First, we demonstrate the \textit{lastChange} usage on a simple example with the comparison to breakpoint debugging. Section 3 presents \textit{lastChange} algorithm. In section 4, we explain the details of the JavaScript prototype implementation. 
%\begin{enumerate}
%/\item The evaluation of \textit{lastChange} implementation.

%\end{enumerate}

%---------------------------------------------------------------------------------------------------
\section{Introductory example}
We illustrate the \textit{lastChange} functionality by a simple example. The example demonstrates a buggy JavaScript code in a html page (Figure ~\ref{fig:js-code}). The page contains a button (line 40) which shows the \texttt{myObject.myProperty} value. It is supposed that whenever user clicks on the button, the value of \texttt{myObject.myProperty} and therefore button's label is increased by one. Once the page is loaded for the first time the button shows \texttt{1} as the initial value of \texttt{myObject.myProperty}. The bug happens when user clicks on the button and instead of \texttt{2}, \texttt{0} appears. 

Once user clicks on the button, \texttt{onClick} function (line 13) is called. This function increases the value of \texttt{myObject.myProperty} by one (line 15) and calls \texttt{updateButton} function which updates button's text to the new value (line 22). To start debugging, the developer sets a abreakpoint on line 22 and once the button is clicked execution is paused at line 22. (Figure ~\ref{fig:example1}) shows the debugger while the execution is paused. The developer can examine the program state at the right panel and as it shows, the \texttt{myObject.myProperty} value at this point is zero which is obviously a wrong value. 

To apply backward search strategy for locating defects, the developer first needs to know the origin to the wrong value. To achieve this goal using breakpoints, the developer should search code to find and set breakpoint at all possible places that \texttt{myObject.myProperty} might get a new value. Even if the developer could sucessfully finish this step (There is no simple way to check it), the next step is monitoring the re-execution and collecting data at breakpoint hits before reaching line 22. All this work is useless if an origin was missed in the first step or if the data was collected in the second step is insufficient.

\begin{figure}[htp]
\begin{verbatim}
1 <html>
...
5   <script type="text/javascript">
6    myObject = {myProperty : 1};
7    myCondition = {value : 1};
...
13   function onClick(){
14     foo();
15     myObject.myProperty++;
16     bar();
17     ...
18     updateButton();
19   }
20   function updateButton(){
21     var myParagraph =
          document.getElementById("myButton");
22     myButton.innerHTML = myObject.myProperty;
23   }   
24   function foo(){
25  	 myCondition.value = oldValue;
26   }  
27   function bar(){ 
28     if (!myCondition.value)
29         myObject.myProperty = 0;
30   }
31  </script> 
...
40  <button id="myButton" onclick="onClick()">
41  	1 
42  </button>
43 </html>
\end{verbatim}
\caption{The buggy JavaScript code in a html page.}
\label{fig:js-code}
\end{figure}


\begin{figure}[htp]
\includegraphics{5-example-points.jpg}
\caption{The examinated points.}
\label{fig:example-points}
\end{figure}

In contrast, we propose a high-level function in debugger, \textit{lastChange}, which provides the answer without demanding anymore effort from the developer. By right clicking on \texttt{myObject.myProperty} in the right panel, the developer can run \textit{lastChange} command (Figure ~\ref{fig:example2}). Debugger considers the paused point at the breakpoint as the reproduction point, and re-executes the buggy execution, pauses at the reproduction point and shows line 29 in a new panel (Figure ~\ref{fig:example3}). The program state debugger shows on the right panel belongs to the past. Therefore the right panel just shows the partially collected data. If the developer needs some values which are not available in the right panel, debugger re-executes another time and collects and shows the requested data. 


\begin{figure*}[htp]

\subfigure[The execution is paused at line 22 by an ordinary breakpoint.]{\label{fig:example1}\includegraphics[width=1.0\textwidth, height=.19\textheight]{1-bp22.jpg}}

\subfigure[\textit{lastChange} command.]{\label{fig:example2}\includegraphics[width=1.0\textwidth, height=.19\textheight]{2-bp22-lastChange.jpg}}

\subfigure[The result of \textit{lastChange} call on \texttt{myObject.myProperty}.]{\label{fig:example3}\includegraphics[width=1.0\textwidth, height=.23\textheight]{3-lastChange.jpg}}

\subfigure[The result of \textit{lastChange} call on \texttt{myCondition.value}.]{\label{fig:example4}\includegraphics[width=1.0\textwidth, height=.23\textheight]{4-lastChange2.jpg}}

\caption{The stages of locating the defect using \textit{lastChange} feature.}
\label{fig:lastChange}
\end{figure*}

Looking at line 29, it seems that something is wrong with \texttt{myCondition.value} which causes line 29 execution. The developer examines \texttt{myCondition.value} and it is \texttt{undefined}. The next step is to know when this property got this value. To do so, the developer runs \textit{lastChange} command on \texttt{myCondition.value} at this point. Debugger re-executes the execution and pauses at the reproduction point and shows line 25-the place \texttt{undefined} value is assigned to \texttt{myCondition.value} (Figure ~\ref{fig:example3}). Now it is clear that the bug occures because \texttt{oldValue} is \texttt{undefined} once execution reaches line 25.

As demonstrated in Figure ~\ref{fig:example-points}, the developer examinated three points of execution. Excepting the first point which debugger pause execution at that point, the second and third points are in the past and are shown by dotted line. These points-the history of the search for the defect-are available through the debugger's interface. On the top of the left panel in Figure ~\ref{fig:example3} there is an opened list which shows all three examinated points. The first one is the breakpoint on line 22, the second one is the point which is the last change of \texttt{myObject.myProperty} before the reaching the breakpoint and finally the last one is the point of execution in which \texttt{myCondition.value} gets the \texttt{undefined} value. Moreover, the source lines related to these points are tagged with red-cycle icons.

Notice that in our example, \textit{lastChange} combines some aspects of breakpoint and of log-based debugging. Like breakpoint debugging, the developer re-executes a live runtime without changing the source and without a special execution environment beyond the debugger. The state of the program memory and the call stack are available at the lastChange point. Like log-based debugging, the program state and the call stack are recorded during program execution. We can't halt the program at \textit{lastChange} because we don't know which is the last one until we return to the original breakpoint. In section 5 we discuss about the possibility of pausing at \textit{lastChange} results.

%---------------------------------------------------------------------------------------------------
\section{\textit{lastChange} Algorithm}

The \textit{lastChange} algorithm is based on program re-execution. Once developer asks for the \textit{lastChange} of a value, debugger re-executes the program, collects limited data and shows the result. \textit{Reproduction point} is the execution point in which first \textit{lastChange} is calld. Like the introductory example, it can be defined by a [conditional] breakpoint. \textit{lastChange} can be called on two different types of values: An object property or a variable value. In JavaScript, every available variable in a frame comes from one of the scopes in the current scope chain. There are five different scope types: global, local, with, catch, closure. We explain each case separately.

\subsection{\textit{lastChange} on object propery}
To simplify the algorithm explanation and avoid technical details, we assume two basic operations and postpone implementation details to the next section. The first opertion is \texttt{objectId()} which gets an object and returns an integer as its identifier. This identifier is unique for every object during one execution. The second operation is \texttt{setPropertyChangeHook()} which gets function as the hook and a string as the property name like \texttt{foo}. Whenever property \texttt{foo} in any object changes the hook will be called. The hook function receives a reference to the owner of \texttt{foo}.

Once developer asks for the last change of \texttt{bar.foo} at the reproduction point, debugger calls \texttt{setPropertyChangeHook()} with \texttt{foo} as the property name and re-executes the program. Whenever \texttt{foo} changes and the hook function is called, debugger first calls \texttt{objectID()} on the \texttt{foo} owner object. Then it stores owner id and stack frame locations. The reason we keep object id instead of a reference to the object itself is to let garbage collector to destroy dead objects. Whenever the execution reaches the reproduction point it looks at the history of \texttt{foo} changes and finds the last \texttt{foo} change with the same object id as \texttt{bar} id at the reproduction point(Figure ~\ref{fig:foo-changes1}). 

\begin{figure}[htp]
\includegraphics[width=.48\textwidth]{6-foo-changes1.jpg}
\caption{The list of property \texttt{foo} changes and \texttt{bar} id which identifies the last change of \texttt{bar.foo} in the history of events.}
\label{fig:foo-changes1}
\end{figure}

\subsection{\textit{lastChange} on variable} 
In JavaScript, every frame has a scope chain and every available variable in the frame comes from one of the scopes in the frame's scope chain. Once developer asks for the last change of variable \texttt{foo} at the reproduction point, debugger first determines the variable's scope as follows: it iterates over the scopes in the scope chain and the first scope which has a variable with the same name is the variable's scope. There are five different scope types: local, global, \texttt{with}, closure, \texttt{catch}. We explain each case separately.


\subsubsection{global and \texttt{with} scopes}
Global scope is the the most outer scope in the scope chain and it also refers as global object. This scope is a regular JavaScript object and therefore every global variable is a property of global object. Similar to global scope, \texttt{with} scopes are also regular objects. A \texttt{with} scope is created by a \texttt{with()} block with an object as the parameter. Every property of passed object is available inside the block as a variable. Debugger treats the case where variable's scope is global or \texttt{with}, like \textit{lastChange} on an object property. 

\subsubsection{local, closure and \texttt{catch} scopes}
local scope refers to the top scope in the scope chain. Clousure scope refers to the scope which is created once a function is defined. Catch scope is the scope created by try-catch blocks and contains the exception variable. These objects are not necessarily regular JavaScript objects. Therefore to track changes to a variable in these scopes we need a new strategy. 


%%% we need a short explanation about scopes
%%% important: think about it: we also need to set a special variable (id) to let us to identify the related functional scope
\begin{figure}[htp]
\begin{verbatim}
10  function parent(arg1){
11    var var1;
12    var1 = 0;
13    function myfun(arg2){
14      var var2;
15      var2 = {...}
16      var1 = var2;
17      function firstChild(arg3){
18        var1 = 1;
19        var2 = 1;
20      }  
21      function secondChild(arg4){
22       	var var1;
23        var1 = 2			      
24        var2 = 2;	
25      }
26    }  
27  }    
\end{verbatim}
\caption{Sample JavaScript code demonstrating closure variables.}
\label{fig:js-closure}
\end{figure}

A variables in these scopes can only be changed by the code which is included in the it's definition block. Therefore it is only enough to take care only about the lines that a variable is assinged a new value. The same fact is also true about \texttt{catch} and local scopes. Debugger puts hooks in on the line the variable is defined and all lines the variable is assigned a new variable (including inner-functions). For example in Figure ~\ref{fig:js-closure}, if the execution is paused at line 24 and the last change of \texttt{var1} is requested by the developer, two hooks on lines 22 and 23 will be set, but if the execution is paused at line 19, four hooks on lines 11, 12, 16, and 18. The reason is that \texttt{var1} in \texttt{firstChild} is a closure variable but in \texttt{secondChild} is a local variable.

\begin{figure}[htp]
\begin{verbatim}

Recursive function: 
 f(){
   var x;
   x++;
   f();
   x; 
   ...
 }

The trace of recursive calls:

  f was called.
    x changes; // A
      f was called;
        x changes;
        f was called 
          x changes; //C
    //C 

\end{verbatim}
\caption{Recursive calls.}
\label{fig:recursive}
\end{figure}

\begin{figure}[htp]
\includegraphics[width=.48\textwidth]{7-foo-changes2.jpg}
\caption{The list of variable \texttt{foo} changes and the \texttt{foo}'s scope id at the reproduction point which identifies the answer.}
\label{fig:foo-changes2}
\end{figure}

Everytime one of the hook hit a new event is inserted to the list of events. To separate irrelvant event from each other, we use a special variable \texttt{\_scopeId}. If the scope has this variable, the value of this variable is used for the scope id otherwise a new id will be generated and \texttt{\_scopeId} variable is created by \texttt{eval()} function in the scope. For example evaluating \texttt{var \_scopeId = 50} will create a variable with name \texttt{\_scopeId} and value \texttt{50} in the local scope. Consider that even in the case of closure scopes, \texttt{\_scopeId} is set when the variable first defined and the variable's scope is still local. Figure ~\ref{fig:recursive} demonstrates one case that scope id is necessary for identifying the right answer. In thiss case the same variable \texttt{x} is changed several times but in different scopes due to recursive calls. Without the scope id calling \textit{lastChange(x)} at point C will return point B while the right answer is point A.
 
\subsection{\textit{lastChange} on \textit{lastChange}}
Once the developer asks for the first \textit{lastChange} all program state is available to the debugger. But the next \textit{lastChange} calls are asked over collected data. For example, debugger needs object id for the second \textit{lastChange} and this data should be collected. To do so, debugger does a data dependency analysis between points which gives a list of items should be collected at every point.

\begin{figure}[htp]
\includegraphics[width=.48\textwidth]{8-lastchange-lastchange.jpg}
\caption{The list of events stored for locating \textit{lastChange} of \texttt{bar.x} (point B) at the reproduction point and \texttt{baz.y} at point B.}
\label{fig:lastchange-lastchange}
\end{figure}
%---------------------------------------------------------------------------------------------------
\section{JavaScript implementation}
The JavaScript prototype is implemented as an extension to Firebug[] javasrcipt debugger. 

\subsection{Execution reproduction}
Altough execution reproduction is basicly should be provided by the developer, we tried to devise some automatic way which reproduces the execution. In the prototype, develooper can choose among three reproduction options. The first one is using a test case which provided by developer. The second one is by record and replay and the third one is local-reproduction.

Two parts should be carefully considered in execution reproduction. First, the initial state should be the same as previous execution. Second, the similar actions and events should be applied to the program during the execution. In the record and replay approach, debugger stores the initial page url and user actions and during replay phase, opens the same url and emulates user actions. 
In the local reproduction approach, debugger calls the funtion on the bottom of stack with the same parameters.

\subsection{Tracking object creation}
Now we discuss two mentioned assumptions. In the standard JavaScript every object has a \texttt{watch()} function which receives two parameters, a property name and another function as a hook. Whenever the property with the given name changes the hook function is called. The issue with using this function is that it can only be applied to available objects. However, at the beginning of re-execution there is no object available. To solve this issue we use a feature in Firefox JavaScript engine, setting a flag gives us the place the object created (e.g., myFile.js, line 24). Though this is a limited data it can help us to get a reference to the object once it is created. Therefore, at the reproduction point, debugger gets the creation location of \texttt{bar} and once the execution reaches to that location it gets a reference to the created object and sets a unique id for it.

\subsection{Data collection}

\subsection{User interface}

\section{Converting \textit{lastChange} result to a conditional breakpoint}
We believe that in many cases it is possible to convert the \textit{lastChange} result to a conditional breakpoint and therefore debugger is able to pause at the point
 instead of showing collected data to the developer. Here, we present an example which shows a case this transformation is not possible.

Figure ~\ref{fig:counter-example} shows a function with processes an array. The array contains numbers except one itme which is \texttt{undefined} and it causes a bug in line 30. There is a call to \texttt{randomPermutation} function in line 11 which randomly permutates the array item. So at every execution the \texttt{undefined} item will be in a new place. Calling \textit{lastChange} on \texttt{x} at the place bug happens, gives a point which shows line 14. Although this point exists at every re-execution but it can not be identified by a conditional breakpoint. 

\begin{figure}[htp]
\begin{verbatim}
10 function randomProcess(array){
11   randomPermutation(array);
12   var x, y;
13   for (var i=0 ; i<array.length ; i++){
14      x = array[i]
...
30      y = x+1;
31   }
32 }
\end{verbatim}
\caption{A counter-example for transforming \textit{lastChange} to a conditional breakpoint.}
\label{fig:counter-example}
\end{figure}
 

%\section{}
%---------------------------------------------------------------------------------------------------
\section{Reproducible Non-deterministic Execution}
Thus far we have not discussed problems caused by mul-tiple threads or other sources of non-deterministic execu-tions. We want to explain why we believe \textit{lastChange} is robust in the practically important case where a bug is reproducible even though the execution may not be deterministic. 
Because \textit{lastChange} require re-execution, we rely on reproducible but not necessarily deterministic execution. A bug is reproducible for a developer when the developer can start from a determined initial state, operate on the program with a list of actions, and reproduce the symptoms of the bug. The details of the execution can change each time we re-execute the buggy program, but the buggy result is the same. The entire query chain reapplies during each execution so the data we show the developer will be internally consistent. The reproducibility of the bug means that the defect is very unlikely to depend on the order of events during the execution.

In this important case of reproducible bugs, \textit{lastChange} are more effective than breakpoints. In the case of logically deterministic program execution, we can use the result from a \textit{lastChange} operation to set a conditional breakpoint then re-execute the program to position the execution trace backwards from our first breakpoint. %(This may be a useful adjunct for Querypoint debuggers to implement, but this backwards motion in the execution logic is not required for Querypoint debugging.) Thus in this case Querypoint debugging can do the same kinds of things as conventional breakpoints just more automatically.

In the case of a non-deterministic program, a \textit{lastChange} is not equivalent to any series of conventional watchpoints or breakpoints. Each time we re-execute a non-deterministic program, the details of execution in-struction order may change. For example, if we record the source code lines every time a conventional watchpoint hits, the record may differ each time we re-execute. Sup-pose we consult one such record and set a breakpoint on the last entry, the apparent lastChange source line. When we re-execute, the breakpoint will hit, but the information we gain may be incorrect: this may not be the lastChange for this particular re-execution. The \textit{lastChange} method records the values we need and the sequence of source lines from all of the watchpoints, then analyzes the record to select the correct lastChange point. The data shown to the developer will be internally consistent, but of course it may change from a previous re-execution, surprising the developer. This is just a signal that the execution is not deterministic. In future we hope to compare queries from successive executions as a tool for learning about non-deterministic executions.

\section{Evaluation}

\section{Related Work}

\textit{lastChange} functionality supports obtaining information about the execution state logically earlier in the control flow. This support resembles a mixture of replay-based and logging-based debugging. Replay-based approaches capture limited data during execution and replay the bug-gy execution to reach past points. In contrast, logging-based approaches collect enough data during execution to relieve developer from re-execution. Replay-based approaches impose much less runtime overhead (about two orders of magnitudes) comparing to logging-based appproches. However, developer has to re-execute the buggy execution several times. \textit{lastChange} functionality collects data on re-execution but this data is limited to the current queries of developer.

Among replay-based debuggers we compare to bdb \cite{Boothe} and reverse watchpoint \cite{Maruyama}.  A bidirectional C debugger, bdb employs a step counter to locate the requested point from the beginning of execution. It relies on deterministic execution replay (i.e., the same sequence of instructions in re-execution) and records the results of non-deterministic system calls and re-injects them into the program when it is replayed. It makes use of checkpoints to reduce the time needed for re-execution.  Reverse watchpoint, is proposed by Maruyama et al., analyses the execution and moves the debugger to the last write access of a selected variable by re-executing the program from the \cite{Maruyama}.  Similar to bdb it relies on deterministic replay and uses a counter to correctly locate a point in the next execution. The main disadvantage of these appraoches is requiring the exactly the same executions. Even one instruction difference between in two executions leads to wrong results. This requirement is that much restricting that after ten years from bdb paper publication, there is no newer implementation of this idea. On the other hand, \textit{lastChange} doesn't require any special feature in the re-rexecution and it is fit to everyday's developers' debugging practice. 

Among logging-based approaches are "omniscient" debuggers ODB\cite{Lewis} and Unstuck[]. Omniscient debuggers have been proposed as a solution for the problems of breakpoint-debugging. Both approaches keep the log history in memory and hence can only record and store the complete history for a short period of time. These debuggers record all the events that occur during the buggy execution and later let the developer to navigate through the obtained execution log. In this approach there is no execution to resume: moving backwards in the log can be similar to moving forwards. Omniscient debuggers suffer from a different set of issues. First, the recording step is time expensive and it should be repeated in case of changes in program. Second, the execution log cannot fully replace the live execution. There are other aspects of execution (e.g., program user interface, file system, database tables, etc.) which are also important in debugging and are not available to the developer in omniscient debuggers. Third, querying collected data (e.g., to restore the program state at a certain point) may not be efficient enough for debugging of realistic programs.

A more scalable approach has been proposed by Pothier et al. \cite{Pothier}. Their back-in-time debugger, TOD, addresses the space problem by storing execution events in a distributed database. Comparing to Omniscient debuggers our approach is lightweight and more flexible. Developer can start debugging just after reproducing bug without a capturing step.  Changing inputs or environment settings and re-executing to investigate the bug works as in conventional breakpoint debuggers.

Two new directions in logging debuggers explore more detailed use of the log and more effective logging approaches. WhyLine\cite{Ko} provides visual interface to collected runtime information and let developer to move  on execution log using queries expressed in terms of the programming objects. WhyLine stores the program user in-terface in addition to program trace and provides answers to why and why not questions to the user. Jive\cite{Czyz} depicts the history of execution by a sequence diagram and lets user to query on events database. Both tools suffer from similar issues with omniscient debuggers; 
%both provide models for extending Querypoint debugging to obtain a better user interface while retaining the flexible conven-tional replay model of debugging.  

A recent work by Lienhard et al.\cite{Lienhard} suggests virtual machine level support for keeping the object flow. It rep-laces every object reference with an alias object which keeps the history of changes to the object reference. In this way, when an object is collected by garbage collector, its track of changes (if it is not referenced by other alias-es) will be also collected. Though this approach incurs less runtime overhead (7 times to 115 times) in comparison to omniscient debuggers, it adds memory overhead. Querypoint debugging uses re-executions to gather infor-mation requested by the developer: the memory overhead depends on the query not the entire program. Moreover, the Lienhard et al. debugger significantly changes the virtual machine, while our approach is a generalization to conditional breakpoints and available debugger infrastructure can be adapted to support it. 
 
\textit{lastChange} functionality does rely on a conventional breakpoint to begin queries, a requirement not shared by full logging solutions.  Here we leverage past experience of developers, but there are also new tools [] to help with this problem in the case of graphical and event based systems.


\section{Conclusions and Future Work}



% We recommend abbrvnat bibliography style.

\bibliographystyle{abbrvnat}

% The bibliography should be embedded for final submission.

\begin{thebibliography}{10}
\softraggedright

\bibitem[Bond(2007)]{Bond}
M.D. Bond, N. Nethercote, S.W. Kent, S.Z. Guyer, and K.S. McKinley. \newblock Tracking bad apples: reporting the origin of null and undefined value errors.
\newblock In \emph{22nd annual ACM SIGPLAN conference on Object-oriented programming, systems, languages, and applications(OOPSLA)},
October, 2007.

\bibitem[Boothe(2000)]{Boothe}
B. Boothe. \newblock Efficient algorithms for bidirectional debugging.
\newblock In \emph{Conference on Programming Language Design and Implementation(PLDI)},
June, 2000.

\bibitem[Czyz(2007)]{Czyz}
J.K. Czyz, and B. Jayaraman. \newblock Declarative and visual debugging in Eclipse.
\newblock In \emph{OOPSLA workshop on eclipse technology eXchange},
October, 2007.

\bibitem[Ko(2008)]{Ko}
A.J. Ko, and B.A. Myers. \newblock Debugging reinvented: asking and answering why and why not questions about program behavior.
\newblock In \emph{30th international conference on Software engineering(ICSE)},
May, 2008.

\bibitem[LaToza(2006)]{LaToza}
T.D. LaToza, G. Venolia, and R. DeLine. \newblock Maintaining mental models: a study of developer work habits
\newblock In \emph{28th international conference on Software engineering(ICSE)},
May, 2006.

\bibitem[Lewis(2003)]{Lewis}
B. Lewis, and M. Ducasse. \newblock Using events to debug Java programs backwards in time.
\newblock In \emph{Companion of the 18th annual ACM SIGPLAN conference on Object-oriented programming, systems, languages, and applications(OOPSLA)},
2003.

\bibitem[Lienhard(2008)]{Lienhard}
A. Lienhard, T. G\^{\i}rba, and O. Nierstrasz. \newblock Practical Object-Oriented Back-in-Time Debugging.
\newblock In \emph{22nd European conference on Object-Oriented Programming(ECOOP)},
July, 2008.

\bibitem[Maruyama(2003)]{Maruyama}
K. Maruyama, and T. Kazutaka. \newblock Debugging with Reverse Watchpoint.
\newblock In \emph{Proceedings of the Third International Conference on Quality Software},
2003.

\bibitem[Pothier(2007)]{Pothier}
G. Pothier, \'{E}. Tanter, and J. Piquer. \newblock Scalable omniscient debugging.
\newblock In \emph{22nd annual ACM SIGPLAN conference on Object-oriented programming, systems, languages, and applications(OOPSLA)},
October, 2007.


\end{thebibliography}

\end{document}
