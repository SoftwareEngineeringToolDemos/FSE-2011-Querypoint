
%%%%%%%%%%%%%%%%%%%%%%% file typeinst.tex %%%%%%%%%%%%%%%%%%%%%%%%%
%
% This is the LaTeX source for the instructions to authors using
% the LaTeX document class 'llncs.cls' for contributions to
% the Lecture Notes in Computer Sciences series.
% http://www.springer.com/lncs       Springer Heidelberg 2006/05/04
%
% It may be used as a template for your own input - copy it
% to a new file with a new name and use it as the basis
% for your article.
%
% NB: the document class 'llncs' has its own and detailed documentation, see
% ftp://ftp.springer.de/data/pubftp/pub/tex/latex/llncs/latex2e/llncsdoc.pdf
%
%%%%%%%%%%%%%%%%%%%%%%%%%%%%%%%%%%%%%%%%%%%%%%%%%%%%%%%%%%%%%%%%%%%


\documentclass[runningheads,a4paper]{llncs}

\usepackage{amssymb}
\setcounter{tocdepth}{3}
\usepackage{graphicx}
\usepackage{subfigure} %[tight,normalsize]
\usepackage{algorithmic}
\usepackage{algorithm}
\usepackage{textcomp}

\usepackage{url}
\urldef{\mailsa}\path|{alfred.hofmann, ursula.barth, ingrid.haas, frank.holzwarth,|
\urldef{\mailsb}\path|anna.kramer, leonie.kunz, christine.reiss, nicole.sator,|
\urldef{\mailsc}\path|erika.siebert-cole, peter.strasser, lncs}@springer.com|    
\newcommand{\keywords}[1]{\par\addvspace\baselineskip
\noindent\keywordname\enspace\ignorespaces#1}

\begin{document}

\mainmatter  % start of an individual contribution

% first the title is needed
\title{Debugging by \textit{lastChange}}

% a short form should be given in case it is too long for the running head
\titlerunning{Debugging by \textit{lastChange}}

% the name(s) of the author(s) follow(s) next
%
% NB: Chinese authors should write their first names(s) in front of
% their surnames. This ensures that the names appear correctly in
% the running heads and the author index.
%



\author{Salman Mirghasemi\inst{1} \and John J. Barton\inst{2} \and Claude Petitpierre\inst{1}}

%
%\authorrunning{Lecture Notes in Computer Science: Authors' Instructions}
% (feature abused for this document to repeat the title also on left hand pages)

% the affiliations are given next; don't give your e-mail address
% unless you accept that it will be published
\institute{Ecole Polytechnique F\'ed\'erale de Lausanne \\
\path|{salman.mirghasemi,claude.petitpierre}@epfl.ch| \\
\and IBM Research - Almaden \\
\path|johnjbarton@johnjbarton.com|}

%
% NB: a more complex sample for affiliations and the mapping to the
% corresponding authors can be found in the file "llncs.dem"
% (search for the string "\mainmatter" where a contribution starts).
% "llncs.dem" accompanies the document class "llncs.cls".
%

\toctitle{Lecture Notes in Computer Science}
\tocauthor{Authors' Instructions}
\maketitle


\begin{abstract}
Developers often seek the origins of wrong values they see in their
debugger. Their search must be backwards in time: the code causing the
wrong value executed before the wrong value appeared. Searching with
breakpoint- or log- based debuggers demands persistence and
significant experience with the application being debugged. 
We introduce a new, practical feature for debuggers called \textit{lastChange}, 
which automatically locates the last point that a variable or an object property 
has been changed. Starting from a 
program halted on a breakpoint, the \textit{lastChange} solution
applies queries to the live program during re-execution, recording the call stack 
and limited program state each time the property value changes.
When the program halts again on the breakpoint, the recorded information 
can be shown to the developer.
As a proof of this concept, we developed \textit{Querypoint}, a prototype
which enhances the popular Firebug JavaScript debugger with
 the \textit{lastChange} feature and studied users applying the prototype to 
some test cases.
 The approach used in implementing 
\textit{lastChange} combines the flexibility of breakpoint debugging
with the expressive power of log-based query debugging.  Contrary to
other replay-based approaches, which require exactly the same
re-executions (deterministic executions), our new approach only requires \textit{bug 
reproducibility}, meaning a test case is available which reproduces the bug and a 
way to halt execution reablity after the reproduction.
  
\keywords{Debugging, Locating Defects, Querypoint, LastChange, Breakpoint,
 Watchpoint, Logging}
\end{abstract}




\section{Introduction}

According to \cite{LaToza}, developers spend about fifty percent of
their time debugging. To fix a bug, developers typically reproduce 
and monitor the buggy execution several times to understand the 
program's unexpected behavior. Trial-and-error, guess-work, and 
analyzing complicated data make debugging difficult and time-consuming. 
Enhancements to debugging to automate 
debugging operations save developer 
time, reducing development costs  and impoving software quality.

A common strategy for locating defects starts from bug symptoms and
works backwards, moving from a point in the program execution where a
value appears to be incorrect back to the point were that value was
set.  Two conventional approaches are used: breakpoint-based and
log-based debugging. Both approaches require tedious steps of
selecting data to be collected, collecting the data, then analyzing
the results. 

In breakpoint debugging, developers select data to be collected by
searching through source files and setting breakpoints. To determine
where a value was set incorrectly, a developer must set
breakpoints at all possible points where the value changes. At
every breakpoint, the developer must determine if the location is in
fact related to the questionable value change then study the complex
debugger user interface and memorize values or manually collect
data. As the number of breakpoint hits increases, the process of
checking the program state, collecting data and resuming the execution
becomes cumbersome.

In log-based debugging, developers select data to be collected by
inserting statements for all points of possible change.  While in
breakpoint-based debugging, the whole program state is available to
developer, in log-based debugging, developer has to decide what data
should be collected when inserts the log statement. It is very common
that the developer has to repeat this step several times due to
insufficient collected data, or to wait a long time because too much
data is recorded. Once the adequate data is collected, it still
requires analyzing and understanding. Developers usually end up in
dealing with long log files and analyzing huge collected data.  None
of these approaches assist developer in finding origins to a wrong
value.

To address these problems, we introduce a new functionality in
debuggers, \textit{lastChange}, which locates the origin of a wrong
value. It uses program re-execution similar to the practice
in using breakpoints and manually inserted log statements. 
Imagine a program execution 
paused on a breakpoint and the developer is
suspicious about the value of a variable or an object property. The
developer queries \textit{lastChange} on the value. The debugger
reproduces the buggy execution and collects data
when the data field changes. Once the execution reaches the same place 
(i.e., the same
breakpoint hit), it pauses the execution, analyzes the collected data
and shows the location of the last change to the developer. The
developer can also examine the program state at the located point of
execution, and continue debugging by more \textit{lastChange} queries
from that point.

As we detail in the Related Work section, numerous papers explore forms 
of automatic logging of program state, augmented with queries on the log. 
This replaces the developer time needed to 
insert statement manually with computer time to write the voluminious logs. 
Our approach uses semi-automatic live-execution queries: user interface 
operations allow the developer to select queries and these queries operate on the live program during re-execution.


We refer to the breakpoint used in \textit{lastChange} as the
\textit{reproduction point}: this is any point where the developer knows
the bug has been reproduced. It is also the point were each \textit{lastChange}
algorithm cycle ends. The ability to identify this reproduction point, i.e.
 a test case and a breakpoint after the bug has appeared, is the only
prerequisite for \textit{lastChange} functionality.  This means
\textit{lastChange} can be implemented on top of a conventional
breakpoint debugger and it does not require a special environment to
create identical, instruction by instruction, re-executions.

Our contribution in this paper is the technique \textit{lastChange},
which locates the last place a value has changed, gathers other values
from that execution point, and allows \textit{lastChange} operations
from that point. The technique builds on existing breakpoint debugger
technology. We demonstrate the feasibility of the approach with %TODO
\textit{Querypoint}, an implementation extending Firebug
JavaScript debugger. \textit{Querypoint} also provides mechanism for
automated bug reproduction, and a novel user interface which
summarizes investigated execution points and collected results.

The rest of the paper is organized as follows. First, we demonstrate
the \textit{lastChange} usage on a simple example with the comparison
to breakpoint debugging. Section 3 presents \textit{lastChange}
algorithm. In section 4, we explain the details of the JavaScript
prototype implementation. We discuss the effect of non-determinism on
the \textit{lastChange} results in section 5. The user study results 
are presented in section 6. %TODO update it according to new Section 6 contains an evaluation of\text{lastChange} functionality.

\section{Introductory example}
\label{sec:introExample}

We illustrate the \textit{lastChange} functionality by a simple
example. The example demonstrates a buggy JavaScript code in a HTML
page (Figure ~\ref{fig:js-code}). The page contains a button (line 40)
showing the value of \texttt{myObject.myProperty}.  When the user
clicks on the button, the \texttt{onClick} function (line 13) is
called. This function increases the value of
\texttt{myObject.myProperty} by one (line 15) and calls
\texttt{updateButton} function which updates button's text to the new
value (line 22).  Once the page is loaded for the first time the
button shows \texttt{1} as the initial value of
\texttt{myObject.myProperty}.  In practice when the user clicks on the
button, \texttt{0} appears instead of \texttt{2}: there is a bug.


Two other functions are called in \texttt{onClick()}, \texttt{foo()}
and \texttt{bar()}. As developers we often encounter function calls
which seem peripheral to our current concern; they may have been added
by another developer, or we may have forgotten their exact properties
or those properties may have changed, and so on. The difference
between what we expect these functions to do, e.g. nothing
interesting, and what they do in practice may cause bugs.



By browsing through the code or other means\cite{Barton},
 the developer determines that the value displayed on the 
button is set at line 22. Since the displayed value is incorrect we know 
the bug occured before we hit this line.
To start debugging, the developer sets a breakpoint
on line 22. Once the button is clicked, execution is paused at line
22. Figure~\ref{fig:example1} shows the Firebug debugger while the
execution is paused. Firebug has several panels (e.g., HTML, CSS,
Script, DOM, etc.) that each demonstrate one aspect of the Web page.
The Script panel contains the list of all loaded source
files and regular debugging facilities such as setting breakpoints and
stepping. To the right of the script panel, the Watch panel shows the program state
where the developer can examine object and variable values. In our case, the
\texttt{myObject.myProperty} value at the paused point is zero. We expected this value to be \texttt{2}.


\begin{figure}[htp]
\begin{verbatim}
1 <html>
...
5   <script type="text/javascript">
6    myObject = {myProperty : 1};
7    myCondition = {value : 1};
...
13   function onClick(){
14     foo();
15     myObject.myProperty++;
16     bar();
17     ...
18     updateButton();
19   }
20   function updateButton(){
21     var myParagraph =
          document.getElementById("myButton");
22     myButton.innerHTML = myObject.myProperty;
23   }   
24   function foo(){
25  	 myCondition.value = oldValue;
26   }  
27   function bar(){ 
28     if (!myCondition.value)
29         myObject.myProperty = 0;
30   }
31  </script> 
...
40  <button id="myButton" onclick="onClick()">
41  	1 
42  </button>
43 </html>
\end{verbatim}
\caption{A Web page containing JavaScript code. Some lines not related to our paper have been elided.}
\label{fig:js-code}
\end{figure}

To apply backward search strategy for locating defects, the developer
first needs to know the origin of the wrong value. To achieve this
goal using breakpoints, the developer should search code to find all possible places that
\texttt{myObject. myProperty} might get a new value and set breakpoint at these locations. However, an
object and property can be accessed and changed through different
names and methods. There is no simple way to identify these aliases or
even their total number.  The developer can make a good guess and set
breakpoints on lines where the property seems to be changed. Then they
re-execute the program and examine the state looking for values that
may lead to the incorrect value observed at line 22. All this work
must be repeated if a new alias is discovered or if some
information related to the buggy result was missed while stopped on
one of the breakpoints.

In contrast, we have added a high-level function in debugger,
\textit{lastChange}, which provides the answer without tedious manual
effort from the developer. By right clicking on
\texttt{myObject.myProperty} in the Watch panel, the developer can run
\textit{lastChange} command (Figure ~\ref{fig:example1}). Debugger
re-executes the program and halts again at the breakpoint on line 22.
However, it shows a new panel, called QP, centered on the source at line 29
(Figure~\ref{fig:example3}), the point of \textit{lastChange}.  To
the right, the TraceData panel shows values of properties of the
program state when it passed through line 29.  These two panels
resemble the Script and Watch panels, but they show data collected by
the debugger at one execution point which is now past: these are
\textit{traces} or \textit{logs} of information collected during the re-execution.

Looking at line 29, it seems that something is wrong with
\texttt{myCondition.value} which causes line 29 execution. The
developer examines \texttt{myCondition.value} and it is
\texttt{undefined}. The next step is to know when this property got
this value. To do so, the developer runs the \textit{lastChange} command
on \texttt{myCondition.value} at this point. This re-executes the
program and pauses at the reproduction point,  showing line 25-the
place \texttt{oldValue} is assigned to
\texttt{myCondition.value}. If the developer asks for \textit{lastChange} on \texttt{oldValue}, 
the debugger can notify the developer that this variable is never assigned a value.
 Now it is clear that the bug occurs because \texttt{oldValue} is
\texttt{undefined} once execution reaches line 25 (Figure~\ref{fig:example4}).

As demonstrated in Figure ~\ref{fig:example-points}, the developer has
examined three points of execution. The first point, the reproduction point,  was the breakpoint set by the developer.
The second and third points preceded the reproduction point in execution sequence.
All three points-the history
of the search for the defect-are available through the debugger's
interface. On the top of the left panel in Figure~\ref{fig:example4}
there is an opened list which shows all three examined points. The
first one is the breakpoint on line 22, the second one is the point
which is when \texttt{myObject.myProperty} changed before
reaching the breakpoint and finally the last one is the point of
execution in which \texttt{myCondition.value} gets the
\texttt{undefined} value. Moreover, the source lines related to these
points are marked with red \textbf{Q} icons.

\begin{figure*}[htp]


\subfigure[A screen shot of the Firebug debugger while running the example code from Fig.~\ref{fig:js-code}. The Script
  panel is selected; it gives access to
  all loaded source files and allows breakpoints to be set on lines. In this
  figure, the execution is paused at line 22 by a regular
  breakpoint. The Watch panel at the right shows the program state at
  the paused
  point. Developer can query \textit{lastChange} on \texttt{myObject.myProperty} by right-clicking on the value of \texttt{myProperty}. ]{\label{fig:example1}\includegraphics[width=1.0\textwidth, height=.17\textheight
  ]{1-bp22.jpg}}


\subfigure[The result of \textit{lastChange} query for
  \texttt{myObject.myProperty}. The left panel, QP, shows the source
  code at the point of \textit{lastChange}; The right panel,
  TraceData, shows the collected data at the
  point.]{\label{fig:example3}\includegraphics[width=1.0\textwidth,% height=.21\textheight
  ]{3-lastChange.jpg}}

\subfigure[The result of \textit{lastChange} query for
  \texttt{myCondition.value}. To evaluate an expression (e.g., oldValue) at this point, developer can %TODO should we remove it?
  enter the expression in the watch box and after re-execution the result is available.
  The opened list on the top of the left panel shows the visited execution points. Clicking on each point in
  the list shows the corresponding code and
  data.]{\label{fig:example4}\includegraphics[width=1.0\textwidth%,height=.25\textheight
    ]{4-lastChange2.jpg}}

\caption{The stages of locating the defect using \textit{lastChange} feature.}
\label{fig:lastChange}
\end{figure*}

\begin{figure}[htp]
\centering 

\includegraphics{5-example-points.jpg}
\caption{The examined points before locating the defect. The arrow represents the logical forward progress of the program. Three actual executions are superimposed on this arrow. All three stop at the reproduction point indicated by circle 1. After the first execution, the developer asks for lastChange as described in \ref{sec:introExample}, yielding information  indicated by circle 2. After the second execution, another lastChange query causes a third execution, yielding information indicated by circle 3.}
\label{fig:example-points}
\end{figure}

Notice that in our example, \textit{lastChange} combines some aspects
of breakpoint and of log-based debugging. Like breakpoint debugging,
the developer re-executes a live runtime without changing the source
and without a special execution environment beyond the debugger. The
state of the program memory and the call stack are available at each
lastChange point. Like log-based debugging, the program state and the
call stack are recorded during program execution. We can't halt the
program at \textit{lastChange} because we don't know which point is the last
one until we return to the original breakpoint. (In section 5 we
discuss cases were it is possible to pause at lines of \textit{lastChange}).


\section{\textit{lastChange} Algorithm}

The \textit{lastChange} algorithm is based on program re-execution of
a program halted on a breakpoint. The algorithm starts when developer
examines the program state at a breakpoint hit and asks for the
\textit{lastChange} of a value. The breakpoint hit becomes the
\textit{reproduction point}. Debugger sets hooks (a callback
function dependent upon the underlying runtime) on all instructions
that might be the result of \textit{lastChange} query. Then the
debugger re-executes the program and every time a hook hits it
checks for a \textit{change event}. In the case of a change, it stores
part of the program state values.  Once the execution reaches the
reproduction point, it analyzes the collected data and shows the
result.  The program state at the execution point of the last change
event is the \textit{lastChange}.


As we described in the preceding section, a \textit{lastChange} query can
be performed on the result of another \textit{lastChange} query. If we
name the reproduction point \textit{R}, we can write the first
\textit{lastChange} in the introductory example in this form:
\textit{lastChange(R, myObject.myProperty)}. It means that this query
is defined at \textit{R}. If we name the result of this query
\textit{L}, we can write the second \textit{lastChange} in this form:
\textit{lastChange(L, myCondition.value)}. In this way, a sequence of
\textit{lastChange} queries with any length can be defined. 

\textit{lastChange} can be called on object property, on a variable value, 
or on the results of a \textit{lastChange}. We explain each case in the following subsections.

\subsection{\textit{lastChange} on object property}
To simplify the algorithm explanation and defer technical details, we
define two basic operations and later we explain the details of these
two operations. The first operation is \texttt{objectId()}: given a
JavaScript object it returns an integer as its identifier. This
identifier is unique to the object during one execution.  By using an
object id instead of an object reference we allow the garbage
collector to reclaim the space for dead objects just as it would in the
absence of the debugger. The second
operation is \texttt{setPropertyChangeHook()}: given a function and a
string, the function is called whenever a property changes and its
name matches the string. For example, if the string is 
\texttt{foo} changes to \texttt{bar.foo} or \texttt{baz.foo} would
call the function.  The callback function receives a reference to the
owner of \texttt{foo}. 

To see how these functions work, suppose the developer asks for the
last change of \texttt{bar.foo} at the reproduction point in a
program. The debugger calls \texttt{setPropertyChangeHook()} with
\texttt{foo} as the property name and re-executes the
program. Whenever \texttt{foo} changes and the callback function is to
be called, debugger first calls \texttt{objectId()} on the
\texttt{foo} owner object. Then it stores this owner id, the stack
frame locations, and other state values in scope at the call point. 
Then the callback returns to continue execution. Thus the query is not 
a breakpoint in the sense of pausing for user interaction, but breakpoint 
technology can be used to implement the query.
Whenever the execution reaches the reproduction point the debugger
looks at the history of \texttt{foo} changes and finds the last
\texttt{foo} change with the same object id as \texttt{bar} id at the
reproduction point. Figure ~\ref{fig:foo-changes1} shows the list of
property \texttt{foo} change events in a hypothetical
execution. \texttt{bar} id at the reproduction point is 1010, so the
last change of \texttt{bar.foo} is the fourth column. 

\begin{figure}[htp]
\centering 
\includegraphics[height=.23\textheight]{6-foo-changes1.jpg} 
\caption{A hypothetical list of  change events for a property \texttt{foo}. 
Each change event adds a column with the id of the object changed, the call stack,
and some program state such as local variable values. 
At the reproduction point  we determine which id  corresponds to 
object \texttt{bar} and read out column 4,  the last
  change of \texttt{bar.foo}. Column 3 is also a change of
  the object we want to study, id 1010, but it is not the last change;
  Column 5 is also a change of a property \texttt{foo} but it is not for
  the object we are interested in.}
\label{fig:foo-changes1}
\end{figure}

\subsection{\textit{lastChange} on variable} 
In JavaScript, every frame has a scope chain and every available
variable in the frame comes from one of the scopes in the frame's
scope chain. Once the developer asks for the last change of a variable with name 
\texttt{foo} at the reproduction point, the debugger first determines the
variable's scope as follows: it iterates over the scopes in the scope
chain and the first scope which has a variable with the same name is
the variable's scope. There are five different scope types: local,
global, \texttt{with}, closure and \texttt{catch}. We explain these
cases in two groups.

\subsubsection{global and \texttt{with} scopes}
Global scope is the most outer scope in the scope chain and it is
also referred to as the global object (the \texttt{window}
object in Web pages). This scope is a regular JavaScript object and therefore every
global variable is a property of global object. Similarly,
 \texttt{with} scopes are also regular JavaScript objects. A \texttt{with}
scope is created by a \texttt{with()} block with an object as the
parameter. Every property of this parameter object is available inside the
block as a variable. \textit{lastChange} treats the case where variable's scope
is global or \texttt{with}, like it does on an object
property.

\subsubsection{local, closure and \texttt{catch} scopes}
Local scope refers to the most nested scope in the scope
chain which contains the local variables. Closure scope refers to the 
scope which is created for a nested function and contains variables 
defined in the outer block. Catch scope is the scope created in the catch
block of try-catch statements and contains the exception
variable. These scopes are not necessarily regular JavaScript
objects. Scope object is usually a transient copy of a native object. 
Therefore, to track changes to a variable in these scopes we
employ a different approach.

Having the scope chain and the source code, we can map every scope to % perhaps needs more explanation, how?
a code block, enclosed the executing code. In
JavaScript, a code block can be identified by the file url and  
the block's first instruction program counter. Given this information, the debugger is
able to recognize the code block in loaded scripts or once it is loaded.
Similar to \textit{lastChange} on object property, we define two basic operations: 
\texttt{scopeId()} and \texttt{setVariableChangeHook()}. The first one, 
given a scope returns an integer as the scope's identifier. The second one, 
given a callback function, a code block and a variable name-which is defined inside the block, the function is called
whenever the variable is changed. 

\begin{figure}[htp]
\begin{verbatim}

  function f(){
    var x = 0;
    x++;
    if (!stop) f();
  }

Sample Trace:

   f()
A  |  x changes, scope 1 
   |  f()
   |  |  x changes, scope 2
   |  |  f()
B  |  |  |  x changes, scope 3 
C  |  ... , scope 1

\end{verbatim}
\caption{A recursive call trace illustrating the scope
  id. The lines in the bottom half of the diagram simulate a trace of the change events for the variable \texttt{x} as the function \texttt{f()} calls itself. Each call creates a scope; eventually when the variable \texttt{stop} is changed by an external process we return from the recursion. The lines marked A, B, and C are discussed in the text.}
\label{fig:recursive}
\end{figure}


Figure ~\ref{fig:recursive} illustrates how the \texttt{ScopeId()} operation separates instances 
of a variable in different scopes having the same name.
If we ask for the last change on \texttt{x} at point labeled \texttt{C} in scope with id 1, we want the change at line \texttt{A} in scope 1, not 
the change at the line marked \texttt{B}, where a variable named \texttt{x} in scope 3 is changed.  


If the developer asks for the last change of variable \texttt{foo} at the 
reproduction point in a program, debugger calls 
\texttt{setVariableChangeHook()} with the variable's defining
block and name as parameters and re-executes the
program. Whenever \texttt{foo} changes and the callback function is to
be called, debugger first calls \texttt{scopeId()} on the
variable's scope. Then it stores this scope id, the stack
frame locations, and other state values in scope at the call point.
Whenever the execution reaches the reproduction point the debugger
looks at the history of \texttt{foo} changes and finds the last
\texttt{foo} change with the same scope id as the variable's scope id
at the reproduction point (Figure ~\ref{fig:foo-changes2}). 

\begin{figure}[htp]
\centering 
\includegraphics[height=.14\textheight]{7-foo-changes2.jpg}
\caption{A hypothetical list of variable \texttt{foo} change events. Each column of the list indicates a change event; for each
change event the scope id returns by \texttt{ScopeId()} is recorded along with call stack and program state information. 
   The last column having the scope id \texttt{foo} at the reproduction point indicates 
  the last change.}
\label{fig:foo-changes2}
\end{figure}


\subsection{\textit{lastChange} on \textit{lastChange}}
The \textit{lastChange} algorithm records changes by id (either object or scope id), then reads out the last change when we
arrive at the reproduction point and discover the id of the requested value.
When we perform \textit{lastChange} based on a previous \textit{lastChange} the query algorithm 
must retain additional information. Consider the following example:
\begin{center}
\textit{
 point A : the reproduction point \\
 point B : lastChange(A, bar.x) \\
 point C : lastChange(B, baz.y) 
 }
 \end{center}
where point A is a breakpoint, point B is the last change of the object property \texttt{bar.x} at point A, and
point C is the last change of the object property \texttt{baz.y} at point B. 
The object referenced by \texttt{baz} changes upon re-execution. Therefore when the developer
asks for the last change of \texttt{baz.y}, we need to track objects named \texttt{baz}
at changes of \texttt{bar.x} and changes to objects named \texttt{y}. Then at the reproduction point
we need to work out which \texttt{baz} the developer wanted, then select the last change of that \texttt{baz.y}.
Figure ~\ref{fig:lastchange-lastchange} illustrates the extra row of data tracking of \texttt{baz} objects 
at \texttt{x} change events 
and how the id values allow the last change of \texttt{baz.y} to be worked out. 
 
In the general case we perform dependency analysis  as outlined in 
Figure~\ref{fig:dependency-analysis} to
create the list of additional data (object id or scope id) to be collected at a change
event. The process can be repeated to cascade \textit{lastChange} arbitrarily deep.

\begin{figure}[htp]
\centering 
%\begin{algorithm}
\begin{algorithmic}

\FOR{$q$ in $lastChange$ queries} 
 \FOR {$p$ is defined at the $q$ result}
   \IF {$p$ is a lastChange on object property} 
     \STATE the property owner id should be stored at $q$ change events. 
   \ELSIF {$p$ is a lastChange on variable} 
     \STATE the variable scope id should be stored at $q$ change events.
	 \ENDIF 
 \ENDFOR 
\ENDFOR

\end{algorithmic}
%\end{algorithm}
\caption{\textit{lastChange} queries dependency analysis.}
\label{fig:dependency-analysis}
\end{figure}

\begin{figure}[htp]
\centering 
\includegraphics[height=.3\textheight]{8-lastchange-lastchange.jpg}
\caption{The list of change events stored for locating point B, the
  \textit{lastChange} of \texttt{bar.x} at the reproduction point, and
  point C, the \textit{lastChange} of \texttt{baz.y} at point B.}
\label{fig:lastchange-lastchange}
\end{figure}



%---------------------------------------------------------------------------------------------------
\section{JavaScript implementation}
To verify the \textit{lastChange} algorithm we
implemented\footnote[1]{http://code.google.com/p/querypoint-debugging} it in an extension to the Firebug
JavaScript debugger\footnote[2]{http://getfirebug.com}. %\cite{Firebug}. %\cite{firebug-version-1.6}. 
Firebug itself is an extension of the Firefox browser.%\cite{Firefox}. %\cite{firefox-version-3.6}
The Firefox JavaScript engine provides a JavaScript debugging interface \cite{JSD} and 
\textit{Querypoint} is developed over this interface. Our prototype implementation defines our four primitive operations in JavaScript code and using techniques which are cumbersome and comparatively slow to execute. However the JavaScript prototype is
easy to explore, change, and share with others for feedback.
  A professionally useful debugger would implement these primitive operations within the JavaScript engine.

\subsection{\texttt{objectId()} operation}
\texttt{objectId(obj)} first checks the argument \texttt{obj} for a property \texttt{\_objectId}.
It returns the value if this property is already defined,
otherwise it generates a new id and sets this property. The value of the id is simply an integer incremented 
for each new \texttt{\_objectId} needed. This use of a regular object property  might change
the program behaviour. For example it will apear as an extra property in 
\texttt{for(property in object)} loops. 
The next version of Firefox\footnote[3]{https://developer.mozilla.org/en/JavaScript/Reference/Global\_Objects\\/Object/defineProperty} provides an alternative via \texttt{defineProperty} a standard JavaScript
function which receives 
an argument specifying whether the property is enumerable or not. By setting the enumerable
field \texttt{false} for \texttt{\_objectId}, this property will not
apear in \texttt{for(property in object)} loops and therefore has no
effect on the program execution. Note that \texttt{\_objectId} is
not set for all objects but only those objects that need an id.

\subsection{\texttt{setPropertyChangeHook()} operation}
The Firefox JavaScript engine supports watching property changes in an
object. Every object has a function \texttt{watch(propName, callback)}
which receives two parameters, a property name and function.  Whenever
the property with the given name changes, the \texttt{callback} function is
called. The hook set by this function remains enabled even if the
property is deleted and defined again. 

For our purposes, the \texttt{watch()} function only covers the case
of global object properties. At the beginning of execution, no object
excepting the global object and its predefined properties is available.  For 
our \textit{lastChange} prototype we created a version of
\texttt{setPropertyChangeHook()}.  The basic strategy is to get a reference to the object just
after its creation, then use \texttt{watch()} function to monitor
property changes in the object. Setting a flag\footnote[4]{DISABLE\_OBJECT\_TRACE defined in jsdIDebuggerService.idl} into the Firefox 
JavaScript engine, we can get file URL and line number for each object
creation (e.g., myFile.js, line 24). We set a hook on this line
and parse the source code to determine which object was created.

The only data we have is the object creation location including the
file url and the line number and the goal is to get a reference to
this object. Although in the most regular cases we have only one
statement and one object creation in a line, there are cases that more
than one object might be created in a line. There is no simple way to
recognize the interesting object among these new objects. So instead
of one object, we monitor all new objects created in the line.

An object might be created by one of these statements: object
initializer (\texttt{\{...\}}), \textit{new} operator (\texttt{new 
constructor()}) or function definition statment (\texttt{function()}). 
By parsing the source code we can recognize the statements that
create an object. The next step is getting a reference to the new object.

The new object can be assigned to an object property or a variable by
an assignment (\texttt{=}). In these cases we keep the assignee statement 
at the left side. The idea is that we create a list of assignee statements 
that the new objects are assigned to. We set a hook on the creation
line. Once the hook hits, we evaluate all assignee statements. Then 
we do stepping(step-over) and after each step we evaluate the
statements. Every statement which has a new value, we consider the new
value-if it is an object-as a new object. For example in Figure
~\ref{fig:objectCreation}(a), if the creation line number is 20, we
have only one statement which creates a new object and it is assigned
to \texttt{x.y}.

The new object can also be set as the property of a parent object by a
colon (\texttt{:}). This case is also treated similar to previous case. The
only difference is that the full path of property from the root parent
in the local scope should be considered as the assignee statement. For
example in Figure~\ref{fig:objectCreation}(b), if the creation line
is 20, we keep \texttt{parentObject.newObject}. In this case stepping
should be continued until the end of line 22 for getting a reference
to the created new object.

There is another case for new objects where they are passed as
arguments to other functions (Figure~\ref{fig:objectCreation}(c)). In these
cases instead of step-over we use step-in and we get the corresponding
argument as a new object.

Although this approach is successful in many ordinary cases, there are some 
cases that a more comprehensive analysis needed for correct behaviour. For example in a case
where the new object is assigned to \texttt{a[++i]} where i is a variable
(Figure ~\ref{fig:objectCreation}(d)), \texttt{a[++i]} is kept in the list
of assignee statements. Obviously, evaluating this statement doesn't return
a reference to the new object. Our prototype implementation does not handle these kinds of unusual cases yet.

Throughout this section we implicitly assumed that the object will be created at 
the same location in the next execution. So, what does happen if it
is not true? Well, once the execution reaches the reproduction point, it
reveals that the object has been created in different location. This time,
prototype re-executes the program considering both locations as possible 
object creation locations.


\begin{figure}[htp]
\begin{verbatim}
(a) Case 1:
20 x.y = new myConstructor();

(b) Case 2:
19  parentObject = {
20   	  newObject : {
21  			x : 5
22  }}

(c) Case 3:
20 myFunction({myProperty:5});

(d) Case 4:
20 a[++i] = new myConstructor();
 
\end{verbatim}
\caption{Examples of different cases in getting references to created JavaScript objects.}
\label{fig:objectCreation}
\end{figure}

\subsection{\texttt{scopeId()} operation}
The prototype sets a hook at the beginning of all code blocks needing
a scope id. The scope id is kept as a variable with name 
\texttt{\_scopeId} in the scope. Whenever the hook is hit,  meaning a new scope
is created, \texttt{\_scopeId} is set by calling JavaScript's dynamic compilation 
function \texttt{eval()}. For example, executing \texttt{eval("var \_scopeId = 10")} creates a
variable with name \texttt{\_scopeId} and value \texttt{10} in the
scope of the \texttt{eval()} call, which is our interesting scope. \texttt{scopeId()}
operation returns the value of \texttt{\_scopeId} in the scope.

\subsection{\texttt{setVariableChangeHook()} operation}
Variables defined in local, closure, and catch scopes are only changed in
their scope; that includes the defining scope and scopes nested whithin them.
For example in Figure ~\ref{fig:js-closure}, variable \texttt{myVar} in line 11 can only be
changed in lines 10 to 23. Therefore, after locating the function in
which the variable is defined, it is enough to parse the code inside
the function block and set a hook on all lines where the variable is
assigned a new value. 
We also set a hook at the first line of the function which is
corresponding to the line where the variable is defined. In Figure
~\ref{fig:js-closure}, if the execution is paused at line 17 and the
last change of \texttt{myVar} is queried, two hooks on lines 16 and 17
will be set, but if the execution is paused at line 20 , four hooks on
lines 11, 12, 14, 20 will be set. These two cases are different due
the fact that \texttt{myVar} in \texttt{firstChild} is a local variable
but in \texttt{secondChild} is a closure variable.

\begin{figure}[htp]
\begin{verbatim}
10  function parent(){
11    var myVar;
12    myVar = ...;
13    function myfun(){
14      myVar = ...;
15      function firstChild(){
16       	var myVar;
17        myVar = ...;
18      }  
19      function secondChild(){
20        myVar = ...;			      
21      }
22    }  
23  }    
\end{verbatim}
\caption{Sample JavaScript code demonstrating local and clousure variables.}
\label{fig:js-closure}
\end{figure}

\subsection{Re-Execution, reproduction point and data collection}
\textit{Querypoint} needs a test case to reproduce the
execution and conditions to correctly recognize the reproduction point. 
Although both elements can be directly provided by developer, \textit{Querypoint}
is also able to automatically create them from the first execution. 

To replay execution, \textit{Querypoint} keeps track of breakpoint hits and single steps. For example, 
if the developer queries \textit{lastChange} at the third hit of breakpoint \textit{b}, in
re-execution, the third hit is recognized as the reproduction point. 

To build a test case from one execution, two elements should be carefully 
considered. First, the initial state should be the same as the initial state
of execution. Second, similar actions and events should be applied to
the program during the execution. \textit{Querypoint} supports two mechanism
for automatic re-execution: callstack-reproduction and record-replay. In callstack reproduction the function from
the earliest frame of the call stack is called with the same parameters. The idea
behind this mechanism is that many bugs in web pages can be reproduced by re-firing
an event like clicking on a button. The record-replay execution uses two phases. In the record phase, it 
stores the initial page url and the events and parameters corresponding to user actions. In the replay phase, it opens
the same url and simulated events as if they were user actions. The local reproduction mechanism provides
shorter re-execution cycles while the record-replay is more accurate about the
initial state.

In addition to the data collected at every change event for identifying the \textit{lastChange}
result, \textit{Querypoint} partially stores values in program state. There is a trade-off between the amount of data collected at every change event and the number of re-executions. If developer asks for 
some values which have not been stored, \textit{Querypoint} re-executes and collects the requested data. 


% Discuss what happens when assumptions changes, like object creation location.

\section{Reproducible Non-deterministic Execution}
We claimed that the only prerequisite for \textit{lastChange} function is bug
reproducibility. A bug is \textit{reproducible} for a developer when the developer can
start from a determined initial state, operate on the program with a
list of actions, and reproduce the symptoms of the bug. The details of
the execution can change each time we re-execute the buggy program,
but the buggy result is the same.  All modern debuggers in wide use that we know 
about rely on reproducible but non-deterministic execution for simple practical reasons: 
developers must reproduce a bug to study it and modern execution environments 
are not deterministic.  

 In this section we discuss non-determinism and \textit{lastChange}. However we want 
to emphasize that when developers have to think about non-determistic execution 
it usually means that bugs are not reproducible. 
The reproducibility of the bug
means that the defect is very unlikely to depend on the order of
events during the execution. Conversely, if the defect depends on the order of execution, 
it will likely fail to reproduce on every execution. By relying on reproducibility but 
not requiring deterministic execution, \textit{lastChange} covers the common case 
developers focus on while not adding addition complexity to the implementation.

\subsection{\textit{lastChange} Result Consistency}
Each time we re-execute a non-deterministic program, 
the details of execution instruction order may change. 
For example, if we record the source code lines every time
a conventional watchpoint hits, the record may differ each time we
re-execute. But \textit{lastChange} does not ask the developer to
consider these single points. Rather it analyzes all of the
change events in an entire execution at the reproduction point where the bug
has already occurred and shows the final result to the developer. 
Neither the data gathering nor the analysis require deterministic execution.

When the developer selects a value and
asks for \textit{lastChange}, the developer is expressing the hypothesis that
the value is related to the defect. Since the defect does not depend on
 exeuction order, the reproducibility of
the bug makes it very unlikely that the \textit{lastChange} will
depend upon order of events during execution.
There can be rare cases where the bug is reproducible but the
value from \textit{lastChange} is not. As in conventional breakpoint
debugging, we can only determine this empirically, by observing
multiple executions with different values. Unlike conventional
breakpoint debugging, implementations of \textit{lastChange} readily
help in detecting such cases: as we work back through a series of
\textit{lastChange} requests we can compare values with previous
executions. Different values on different executions will signal that
the execution is not deterministic. 

Consider the example in Figure~\ref{fig:nondeterministic-values}. 
Assume that at the first exeuction, developers sees that \texttt{a} is \texttt{true} and asks
for the \textit{lastChange} of \texttt{a} at the breakpoint. In the re-exeuction, \texttt{a} is \texttt{false},
and \texttt{b} is \texttt{true} at the reproduction point. It means that \texttt{lastChange}
result shows line 10, which is a correct answer for this execution but not for the previous one.
When user asks for the \textit{lastChange} on variable \texttt{a}, debugger stores \texttt{a} value and compares
it to the \texttt{a} value in the next executions. So if this value is different, debuggers inform the developer
that the \textit{lastChange} query is made on a different value. In cases that the value is not a primitive but an object,
the primitive properties of the object examined by the developer are stored and compared.

\begin{figure}[htp]
\begin{verbatim}
10 a = false;
11 b = false;
...
20 if ( random() ) { a = true; } else { b = true; }
...
30 if (a || b) { bug();  /* breakpoint */ }
\end{verbatim}
\caption{\textit{lastChange} on non-deterministic values.}
\label{fig:nondeterministic-values}
\end{figure}

\subsection{Combination of \textit{lastChange} and breakpoint debugging}
\label{sec:pausing}
Using \textit{lastChange} a developer can work backwards on the flow of data, but sometimes bugs are more obvious when
we watch control flow forwards. Consider for example, 
\begin{verbatim}
44              vector.r = Math.floor(Math.random()*5)*6;
45              if (vector.r !== 0 && vector.r < 30);
46                   return vector;
\end{verbatim}
where \textit{lastChange} shows us that \texttt{vector.r} is zero at line 44. In our user studies, developers
wanted to single step forward from line 44. If they could do this, they may be surprised to see line 46 execute 
even if \texttt{vector.r} is zero, directing their attention to line 45 where they can discover the errant semicolon.
However, a \textit{lastChange} is just a query result, not a breakpoint. Under what conditions can we 
cause the debugger to stop at the \textit{lastChange} point?

In the case that the execution is deterministic, the \textit{lastChange}
event index can be used as an index for a conditional breakpoint. 
Moreover, this works in a non-deterministic execution if non-determinism
has no effect on the event index.  The debugger can easily set this conditional breakpoint
and reexecute the program. If the stream of change events differs in this re-execution the
developer can be warned that the conditional breakpoint may not be the \textit{lastChange} point.
However, if the stream of events is the same, we do not know that the 
conditional breakpoint matches the last 
change, this fact  can only verify  at the reproduction point. These theoretical concerns are not
likely to cause significant practical problems: if the value at the conditional breakpoint is not suspcious, 
then the developer will know to simply return to \textit{lastChange} or move on to another tactic to find the bug.

There are cases where event indexes fail, but a developer can use knowledge gained from \textit{lastChange}
to formulate a conditional breakpoint.
Consider the example in Figure~\ref{fig:counter-example}. It shows a function which processes an
array. The array contains numbers except one item which is
\texttt{undefined} and it causes a bug in line 30. There is a call to
\texttt{randomPermutation} function in line 11 which randomly
permutes the array item. So at every execution the
\texttt{undefined} item will be in a new place. Calling
\textit{lastChange} on \texttt{x} at the place the bug happens, gives a
point which shows line 14. Although this point exists at every
re-execution but it can not be identified by an event index. A conditional breakpoint on 
the value of the item would succeed and the developer can reason from the 
\textit{lastChange} results to create the condition. 

\begin{figure}[htp]
\begin{verbatim}
10 function randomProcess(array){
11   randomPermutation(array);
12   var x, y;
13   for (var i=0 ; i<array.length ; i++){
14      x = array[i]
...
30      y = x+1;
31   }
32 }
\end{verbatim}
\caption{A counter-example for transforming \textit{lastChange} to a conditional breakpoint.}
\label{fig:counter-example}
\end{figure}


\section{User Study}
We supplied four experienced Javascript developers with our prototype lastChange in an 
extended Firebug debugger\footnote[5]{http://ltiwww.epfl.ch/\texttildelow mirghase/lastchange-userstudy}. Following a tutorial and a practice case, we observed as they 
applied both conventional breakpoint and \textit{lastChange} on two small programs we 
provided. (Our prototype did not have a user interface to support Sec.~\ref{sec:pausing}).
All four developers successfully applied \textit{lastChange} to the test programs 
and understood how it could help debugging. 

We designed the example programs to require two steps of \textit{lastChange}; 
all four users took more steps (9, 13, 12, and more than 30) to find the defect location with breakpoints.  
To find locations for breakpoints two users scrolled through the source, 
another searched the text, a third set a lot of breakpoints to get an understand of the control flow. 
Based on our own experience we expect these strategies represent the kinds of approaches 
developers have available. These operations are time consuming and tedious: all four users 
found the defect location with just two \textit{lastChange} operations. Based on our measurements,
this makes \textit{lastChange} about three times faster than breakpoints for the navigation from 
the reproduction point to the defect.
(We recognize that these programs were designed to highlight lastChange and many kinds of debugging 
issues have been hidden by the design of our tests). Our results show that, when a defect 
relates to incorrect values and a developer recognizes this, then the operational 
mechanics of lastChange lead to the defect much more quickly than breakpoints.

Our observation and discussions with users also brought out several important issues and improvements
for our user interface. Perhaps the most important and challenging improvement would be better integration 
with breakpoint debugging. Our implementation put the results of \textit{lastChange} in a similar but different
view from breakpoint debugging. This focuses attention on value changes, but it makes studying control 
flow more difficult: we don't support single stepping from a \textit{lastChange} result in our user interface. In our next iteration we 
plan to merge the query and breakpoint results and support pausing as described in Sec.~\ref{sec:pausing}.

\section{Discussion}

We have presented the \textit{lastChange} algorithm and described our
prototype implementation. Our goal is practical improvements in debugging. To achieve our goal we need practical JavaScript engines
to add new debug primitives so that developers in the field can use our new technique. This paper is
one step to convince implementers to enable \textit{lastChange}.
Thus we summarize here our arguments that \textit{lastChange} should be supported.

The key ingredients in our argument are: 
\begin{enumerate}
   \item developers need an operation like \textit{lastChange}, 
   \item developers can learn to use \textit{lastChange}, 
   \item practical implementations are feasible with modest
invested development time,
   \item in most cases \textit{lastChange} will be much faster than current alternatives,
   \item the worst cases are not more common or more painful than
alternatives.
	 \item the \textit{lastChange} approach can be generalized to other queries.
\end{enumerate}

%TODOWe have also developed a prototype implementation of the debugger control part of lastChange for Java. The main differences (one or two sentences). We will report on this implementation in a later publication
\subsection{Developers need an operation like \protect\textit{lastChange} }

We have shown how \textit{lastChange} identifies the prior point in
program execution where a program state value changes. Is it something
the developer needs to do?  Since ultimately programs are just
transformation of state values, debugging is ultimately backtracking
to find defects in program state change\cite{Zeller}.  When a developer halts a
program on a breakpoint they have three kinds of information: a model
of the program in their mind, a call stack showing what part of the
program is halted, and the state at that point (a combination of the
debugger's view of the program internal state and the output of the
program up to this point of execution). If the model does not match
the call stack or if no value in the viewable program state aligns
with the developers model for this point in execution, the developer
will seek another view by changing the breakpoint or the input
data. If a value appears incorrect, they may have a flash of insight
and know the defect. Otherwise they need to figure out what operation
causes the incorrect value. Here \textit{lastChange} takes over and addresses a key
part of the debugging process.

\subsection{Developers can learn to use \protect\texttt{lastChange}}

If \texttt{lastChange} could be useful, can developers figure out how
to use it? While our user study was small , our prototype demonstrates that
\textit{lastChange} can be easily activated by operations on the
graphical representation of erroneous data in a debugger and the 
results can be interpreted by users. We designed the
user interface for the results from \textit{lastChange} to resemble
the results from a breakpoint, with the source code of the change
point highlighted and the state of the program presented the way state
is presented from a breakpoint. Based on this user interface, we
believe developers can start to use \textit{lastChange} with minimal
training. Our users may not be representative, but the key point here
is that the learning curve is small compared with the challenge of learning 
breakpoint debugging in the first place.

While \textit{lastChange} is not a breakpoint, 
all of the assumptions developers already have for breakpoints hold
for \textit{lastChange}. In particular the re-execution is just the
same operation developers use to debug with breakpoints. 
In our own experience, \textit{lastChange} dramatically reduces the work in
setting and removing breakpoints, so work to cause re-execution  become noticable.
Therefore automatic mechanisms
for re-execution will be corresponding more valuable for debugging
with \textit{lastChange}.  (Our \textit{Querypoint} prototype  
implements a simple form of automatic 
re-execution for both breakpoint and \textit{lastChange}).


\subsection{Practical implementations are feasible}

We have described our prototype all-JavaScript implementation in Sec. 4. It is adequate for 
exploring the ideas and may even be usable in production. However significant 
improvments can be made. A fully usable
implementation would require access to the object id at the point of
object creation and \texttt{setPropertyChan geHook()}. Many
object-oriented runtimes provide object identifiers and 
provide access to object creation directly or by bytecode instrumentation\cite{JPDA}. 
In the particular case of the Firefox Web browser we used for our prototype, the primary barrier
to practical implementation would likely be integration with the
increasingly sophisticated just in time compilers. In the case of
tracking variable changes, \texttt{setVariableChangeHook()} can be implemented
similar to \texttt{setPropertyChangeHook()} if JavaScript scopes-like regular 
JavaScript objects-support \texttt{watch()} function.

\subsection{In most cases \protect\texttt{lastChange} will be much faster than current alternatives}

When developers try \textit{lastChange}, will they get results fast
enough to benefit? While we only used our prototype on toy programs,
\textit{lastChange} seemed as fast as breakpoint re-execution.
But what can we say about realistic programs?

Recall that we
insert additional code through debugger callbacks, then re-execute the
program. The additional code we insert is proportional (in our
JavaScript algorithm) to 1) the number of places a property or
variable with a given name is changed, 2) the number of places objects
are created. The overhead for each execution at a change event
 depends upon the amount of data we store for each change event; 
the overhead for object creation could be small since we only need to 
determine if the object is one we need to watch.  

We implemented a
simple and effective mechanism allowing developers to control 
overhead: we stored data to a depth of two property accesses 
(e.g. foo.bar.x), then placed a 're-run' button 
in the user interface at the third level. If developers reached the 
third level they can see more data at the cost of one more replay.

For comparison we should use the practical alternative: developers
setting breakpoints. For the vast majority of programs, a developer
will take much more time to set one breakpoint than
\textit{lastChange} would add. But typically the developer may not
guess the point of last change. They must then ponder another
breakpoint and re-execute. Unlike the re-execution we use to 
optimize performance in \textit{lastChange}, 
developers still will not know if the breakpoint they hit is the point of last change
 or even if it is related to the change at all.

We could also compare to solutions based on logging or tracing. Manual
logging has very high overhead: the developer must add code, debug
that added code, then analyze the log. (To be fair, the log can become
a permanent debugging aid.) Automatic logging as we discuss in 
section \ref{sec:relatedWork} causes about one or two orders of magnitude slow down as well
as requiring a completely different set of development tools. 

\subsection{The worst cases are not more common or more painful than alternatives}

Finally we consider the worst cases: what about code that changes
objects in long running loops? Every time through the loop we incur
the call back overhead; if the loop itself has relatively little code
the overhead could be very large; if the loop computation is a
significant fraction of the full program, the slowdown would be
enormous. 

These are difficult issues for other techniques as well:  breakpoints are not feasible in these cases and logging
becomes unwieldy. Since automatic logging solutions are highly tuned,
their overhead in this case will likely be much less than
\textit{lastChange}. On the other hand, \textit{lastChange} integrates
with an interactive debugger and detecting that we are in a high
overhead loop is simply a matter of checking our internal counter. In
such unusual cases we may simply offer the developer the option of
studying the loop code then omitting it from \textit{lastChange}
operations. Developers face this issue with any debugger today: occasionally a
debugger causes too much overhead to be useful for debugging. 
The users of \textit{lastChange} may encounter these kinds of cases, but 
that is the nature of debugging: you need different tools for 
different kinds of problems.

\subsection{Generalizations}
Our \textit{lastChange} algorithm can be viewed as a particular interface to a general facility. 
The general facility replays execution, querys the runtime at points of interest during execution,
and analyzes the result at the reproduction point. We have selected one kind of query and analysis
that can be easily integrated in existing debuggers and explained to developers, providing automation of 
the problem of finding the point of last change. We believe other
kinds of queries and analysis can be invented and integrated to automate other aspects of debugging.
Moreover, the query history analysis used to test for non-determinism here may also be the basis for
more effective user interfaces to debugging sessions. The series of cascaded \textit{lastChange} queries becomes 
logical path followed by the developer seeking the defect and this path may be useful if presented in effective user interfaces.

\section{Related Work}
\label{sec:relatedWork}

The algorithm we presented in this paper obtain information about
the execution state logically earlier in the control flow by querying 
the executioin state during replay using the technology of 
conventional breakpoint debuggers.  This
approach resembles the operational model of replay-based debugging 
and the query approach of logging-based
debugging.  Replay-based approaches capture limited data during
execution and replay the buggy execution to reach past points. In
contrast, logging-based approaches collect enough data during
execution to relieve developer from re-execution, then query the data to 
inform the developer. Replay-based
approaches impose much less runtime overhead (about two orders of
magnitudes) comparing to logging-based approaches. However, developer
has to re-execute the buggy execution several
times. \textit{lastChange} collects data on re-execution by queries
selected by developer interaction with the debugger. Therefore we have 
the selectivity of the replay-based approaches that improves performance, 
but we have the flexibilty of the queries so we do not require deterministic replay.

Among replay-based debuggers we compare to bdb \cite{Boothe} and
reverse watchpoint \cite{Maruyama}.  A bidirectional C debugger, bdb
employs a step counter to locate the requested point from the
beginning of execution. It relies on deterministic execution replay
(i.e., the same sequence of instructions in re-execution) and records
the results of non-deterministic system calls and re-injects them into
the program when it is replayed. It makes use of checkpoints to reduce
the time needed for re-execution.  Reverse watchpoint, is proposed by
Maruyama et al., analyses the execution and moves the debugger to the
last write access of a selected variable by re-executing the program
from the beginning\cite{Maruyama}.  Similar to bdb it relies on deterministic
replay and uses a counter to correctly locate a point in the next
execution. The main disadvantage of these approaches is that they require 
exactly the same executions. Even one instruction difference between
two executions leads to wrong results. On the other hand,
\textit{lastChange} doesn't require any special feature in the
re-execution and it is fit to  developers' everyday debugging
practice.

Among logging-based approaches are \textit{omniscient} debuggers
ODB\cite{Lewis} and Unstuck\cite{Hofer}. Both
approaches keep the log history in memory and hence can only record
and store the complete history for a short period of time. These
debuggers record all the events that occur during the buggy execution
and later let the developer to navigate through the obtained execution
log. In this approach there is no execution to resume: moving
backwards in the log can be similar to moving forwards. Omniscient
debuggers suffer from a different set of issues. First, the recording
step is time expensive and it should be repeated in case of changes in
program. Second, the execution log cannot fully replace the live
execution. There are other aspects of execution (e.g., program user
interface, file system, database tables, etc.) which are also
important in debugging and are not available to the developer in
omniscient debuggers. Third, querying collected data (e.g., to restore
the program state at a certain point) may not be efficient enough for
debugging of realistic programs.

A more scalable approach has been proposed by Pothier et
al. \cite{Pothier}. Their back-in-time debugger, TOD, addresses the
space problem by storing execution events in a distributed
database. Comparing to Omniscient debuggers our approach is
lightweight and more flexible. Developer can start debugging just
after reproducing bug without a capturing step.  Changing inputs or
environment settings and re-executing to investigate the bug works as
in conventional breakpoint debuggers.

A recent work by Lienhard et al.\cite{Lienhard} suggests virtual
machine level support for keeping the object flow. It replaces every
object reference with an alias object which keeps the history of
changes to the object reference. In this way, when an object is
collected by garbage collector, its track of changes (if it is not
referenced by other aliases) will be also collected. Though this
approach incurs less runtime overhead in
comparison to omniscient debuggers (7 times to 115 times), it adds memory
overhead. 

Origin tracking of \texttt{undefined} and \texttt{null} values employing \textit{value piggybacking} technique proposed by
Bond et al. \cite{Bond}. This approach has two main limitations comparing to \textit{lastChange}.
First, it is limited to \texttt{undefined} and \texttt{null} values. Second, this approach dose not return the last change
of a \texttt{null} variable but the first place that the \texttt{null} value is originated.

Two new directions in logging debuggers explore more detailed use of
the log and more effective logging approaches. WhyLine\cite{Ko}
provides visual interface to collected runtime information and lets
developer moves on execution log using queries expressed in terms of
the programming objects. WhyLine stores the program user interface in
addition to program trace and provides answers to why and why not
questions to the user. Jive\cite{Czyz} depicts the history of
execution by a sequence diagram and lets user to query on events
database. Both tools suffer from the requirement of gathering tracing information before their unique capabilities can be used.
We imagine that the runtime queries we use in \textit{lastChange}  may be used to gather data incrementally for these kinds of debugging approaches.



%Querypoint debugging uses re-executions to gather infor-mation requested by the developer: the memory overhead depends on the query not the entire program. Moreover, the Lienhard et al. debugger significantly changes the virtual machine, while our approach is a generalization to conditional breakpoints and available debugger infrastructure can be adapted to support it.
 
%\textit{lastChange} functionality does rely on a conventional breakpoint to begin queries, a requirement not shared by full logging solutions.  Here we leverage past experience of developers, but there are also new tools [] to help with this problem in the case of graphical and event based systems.


\section{Conclusion}
\textit{lastChange} provides critical information for debugging programs: the location and state at the point where a questionable value was assigned. It builds upon existing technology and developer experience making it a practical solution for implementers. Our prototype demonstrates the feasibility of \textit{lastChange} and its user interface hints at the potential this approach can have in organizing the debugging experience. 










%Leo (the non-det values case), other readers
\subsubsection*{Acknowledgments.} TODO


\begin{thebibliography}{16}

\bibitem{Barton}%[Bond(2010)]
J.J. Barton, and J. Odvarko. \newblock Dynamic and graphical web page breakpoints.
\newblock In \emph{Conference on World Wide Web(WWW)},
April, 2010.

\bibitem{Bond}%[Bond(2007)]
M.D. Bond, N. Nethercote, S.W. Kent, S.Z. Guyer, and K.S. McKinley. \newblock Tracking bad apples: reporting the origin of null and undefined value errors.
\newblock In \emph{22nd annual ACM SIGPLAN conference on Object-oriented programming, systems, languages, and applications(OOPSLA)},
October, 2007.

\bibitem{Bond2}%[Bond(2010)]
M.D. Bond, G.Z. Baker, S.Z. Guyer, and Z. Samuel. \newblock Breadcrumbs: efficient context sensitivity for dynamic bug detection analyses.
\newblock In \emph{Conference on Programming Language Design and Implementation(PLDI)},
June, 2010.

\bibitem{Boothe}%[Boothe(2000)]
B. Boothe. \newblock Efficient algorithms for bidirectional debugging.
\newblock In \emph{Conference on Programming Language Design and Implementation(PLDI)},
June, 2000.

\bibitem{Czyz}%[Czyz(2007)]
J.K. Czyz, and B. Jayaraman. \newblock Declarative and visual debugging in Eclipse.
\newblock In \emph{OOPSLA workshop on eclipse technology eXchange},
October, 2007.

%\bibitem{Firebug}%[Firebug(2010)]
%Firebug. \newblock http://getfirebug.com.

%\bibitem{Firefox}%[Firefox(2010)]
%Firefox. \newblock http://www.mozilla.com.

\bibitem{Hofer}%[Hofer(2006)]
C. Hofer, M. Denker, and S. Ducasse. \newblock Implementing a backward-in-time debugger.
\newblock In Proceedings of\emph{NODe'06},
volume P-88, pages 17-32. Lecture Notes in Informatics, 2006.

\bibitem{JPDA}%[JSD(2010)]
Java Platform Debugger Architecture. \newblock http://java.sun.com/javase/technologies/ core/toolsapis/jpda.

\bibitem{Ko}%[Ko(2008)]
A.J. Ko, and B.A. Myers. \newblock Debugging reinvented: asking and answering why and why not questions about program behavior.
\newblock In \emph{30th international conference on Software engineering(ICSE)},
May, 2008.

\bibitem{LaToza}%[LaToza(2006)]
T.D. LaToza, G. Venolia, and R. DeLine. \newblock Maintaining mental models: a study of developer work habits
\newblock In \emph{28th international conference on Software engineering(ICSE)},
May, 2006.

\bibitem{Lewis}%[Lewis(2003)]
B. Lewis, and M. Ducasse. \newblock Using events to debug Java programs backwards in time.
\newblock In \emph{Companion of the 18th annual ACM SIGPLAN conference on Object-oriented programming, systems, languages, and applications(OOPSLA)},
2003.

\bibitem{Lienhard}%[Lienhard(2008)]
A. Lienhard, T. G\^{\i}rba, and O. Nierstrasz. \newblock Practical Object-Oriented Back-in-Time Debugging.
\newblock In \emph{22nd European conference on Object-Oriented Programming(ECOOP)},
July, 2008.

\bibitem{Maruyama}%[Maruyama(2003)]
K. Maruyama, and T. Kazutaka. \newblock Debugging with Reverse Watchpoint.
\newblock In \emph{Proceedings of the Third International Conference on Quality Software},
2003.

\bibitem{JSD}%[JSD(2010)]
Mozila JavaScript Debugging Interface. \newblock http://www.mozilla.org/js/jsd.

\bibitem{Pothier}%[Pothier(2007)]
G. Pothier, \'{E}. Tanter, and J. Piquer. \newblock Scalable omniscient debugging.
\newblock In \emph{22nd annual ACM SIGPLAN conference on Object-oriented programming, systems, languages, and applications(OOPSLA)},
October, 2007.

\bibitem{Zeller}%[Pothier(2007)]
A. Zeller. \newblock Why programs fail: A gauide to systematic debugging.
Morgan Kaufmann (2005)

\end{thebibliography}

\end{document}
